\documentclass[11pt]{article}
\usepackage[margin=1in]{geometry} 
\usepackage{amsmath,amsthm,amssymb,amsfonts}
\usepackage[utf8]{inputenc}
\usepackage[T1]{fontenc}
\usepackage{microtype}
\usepackage{mathpazo}
\usepackage{euler}
\usepackage{xcolor}
\usepackage{tikz}
\usepackage{tikz-cd}
\usetikzlibrary{arrows}
\usetikzlibrary{matrix}
\usepackage{fancyhdr}
\pagestyle{fancy}
\usepackage{enumitem}

\newcommand{\N}{\mathbb{N}}
\newcommand{\Z}{\mathbb{Z}}
\newcommand{\Q}{\mathbb{Q}}
\newcommand{\R}{\mathbb{R}}
\newcommand{\C}{\mathbb{C}}
\newcommand{\Ss}{\mathbb{S}}
\newcommand{\M}[2]{\mathsf{M}_{#1}#2}
\newcommand{\im}{\operatorname{im}}
\newcommand{\eps}{\varepsilon}
\newcommand{\dpart}[2]{\frac{\partial#1}{\partial#2}}
\newcommand{\nat}[1]{[\![#1]\!]}
\newcommand{\natzero}[1]{\nat{#1}_0}
\newcommand{\adj}[1]{\operatorname{adj}(#1)}
\newcommand{\ip}[2]{\langle #1 , #2 \rangle}
\newcommand{\ol}{\overline}
\newcommand{\hook}[3]{\frac{\partial}{\partial x_{#1}}\Big\rvert_{#2}^{#3}}
\usepackage{rotating}
\newcommand*{\isoarrow}[1]{\arrow[#1,"\rotatebox{90}{\LARGE{\(\sim\)}}"
]}
\renewcommand{\d}{\operatorname{d}}
\definecolor{color}{RGB}{191, 0, 124}
\newcommand{\paint}[1]{\color{color}{#1}}

\renewcommand*{\proofname}{\paint{Demostraci\'on}}
\newenvironment{theorem}[2][Teorema]{\begin{trivlist}
\item[\hskip \labelsep \paint{{\bfseries #1}}\hskip \labelsep {\bfseries #2.}]}{\end{trivlist}}
\newenvironment{lemma}[2][Lema]{\begin{trivlist}
\item[\hskip \labelsep \paint{{\bfseries #1}}\hskip \labelsep {\bfseries #2.}]}{\end{trivlist}}
\newenvironment{exercise}[2][Ejercicio]{\begin{trivlist}
\item[\hskip \labelsep \paint{{\bfseries #1}}\hskip \labelsep {\bfseries #2.}]}{\end{trivlist}}
\newenvironment{obs}[2][Observaci\'on]{\begin{trivlist}
\item[\hskip \labelsep \paint{{\bfseries #1.}}]}{\end{trivlist}}
\newenvironment{reflection}[2][Resoluci\']{\begin{trivlist}
\item[\hskip \labelsep {\bfseries #1}\hskip \labelsep {\bfseries #2.}]}{\end{trivlist}}
\newenvironment{proposition}[2][Proposici\'on]{\begin{trivlist}
\item[\hskip \labelsep \paint{{\bfseries #1}}\hskip \labelsep {\bfseries #2.}]}{\end{trivlist}}
\newenvironment{corollary}[2][Corolario]{\begin{trivlist}
\item[\hskip \labelsep {\bfseries #1}\hskip \labelsep {\bfseries #2.}]}{\end{trivlist}}

%-----------------------

\title{
\LARGE{\paint{Geometr\'ia Diferencial}}
\\
\vspace{0.5pt}
\small{\paint{Ejercicios para Entregar - Pr\'actica 4}}
}
\author{\paint{Guido Arnone}}
\date{}
\lhead{Guido Arnone}
\rhead{Pr\'actica 4}

\begin{document}

\maketitle

\begin{center}
\paint{\large{Sobre los Ejercicios}}
\end{center}
\begin{center}
Elegí los ejercicios $\paint{(6)}$, $\paint{(8)}$ y el inciso $\paint{(d)}$ del ejercicio $\paint{(11)}$.
\end{center}
\begin{center}
$\paint{
\rule{400pt}{0.5pt}
}$
\vspace{35pt}
\end{center}

\begin{exercise}{6} Si $M$ y $N$ son variedades no vac\'i­as, entonces $M\times N$ es
orientable si y s\'olo si $M$ y $N$ lo son.
\end{exercise}
\begin{proof} Veamos ambas implicaciones
\begin{itemize}[listparindent = \parindent]
\item[($\Rightarrow$)] Supongamos que $M \times N$ es orientable.
\item[($\Leftarrow$)] Ahora supongamos que tanto $M$ como $N$ son orientables.
\end{itemize}
\end{proof}

\begin{exercise}{8} 
\hspace{1pt}
\begin{itemize}[listparindent = \parindent]
\item[a)] Si $f:M\to N$ es una funci\'on diferenciable entre variedades, probar que los pull-backs $f^*:\Omega^k(N)\to\Omega^k(M)$ son tales que
\begin{itemize}[listparindent = \parindent]
\item[(i)] $f^*(\omega_1+\omega_1) = f^*(\omega_1) + f^*(\omega_2)$
\item[(ii)] $f^*(h\cdot \omega_1) = h\circ f\cdot f^*(\omega_1)$ 
\item[(iii)] $f^*(\omega_1\wedge\omega_2) = f^*(\omega_1)\wedge f^*(\omega_2)$
\end{itemize}
para cada $\omega_1$,~$\omega_2\in\Omega^\bullet(N)$ y $h\in C^\infty(N)$.

\item[b)] Si $U$ y $V$ son abiertos de $\R^n$ y $f:U\to V$ es diferenciable,
entonces
\begin{align*}
  f^*(\d x_i)=\sum_{k=1}^n\frac{\partial f_i}{\partial x_k}\d x_k
\end{align*}
y
\begin{align*}
  f^*(g\cdot\d x_1\wedge\cdots\wedge\d x_n)
        = g\circ f
                \cdot\det\Bigl(\frac{\partial f_i}{\partial x_j}\Bigr)_{i,j}
                \cdot \d x_1\wedge 	\cdots\wedge\d x_n
\end{align*}
para cada $i\in\{1,\dots,n\}$ y cada $g\in C^\infty(V)$.
\end{itemize}
\end{exercise}
\begin{proof} Hacemos cada inciso por separado.
\begin{itemize}[listparindent = \parindent]
\item[(a)] Si $\omega_1,\omega_2 \in \Omega^k(N)$ y $v_1, \dots, v_n \in T_pN \subset TN$, entonces
\begin{align*}
f^*(\omega_1+\omega_2)_p(v_1, \dots, v_k) &= (\omega_1+\omega_2)_{f(p)}(f_{\ast,p}(v_1), \dots, f_{\ast,p}(v_k))\\
&= {\omega_1}_{f(p)}(f_{\ast,p}(v_1), \dots, f_{\ast,p}(v_k)) + {\omega_2}_{f(p)}(f_{\ast,p}(v_1), \dots, f_{\ast,p}(v_k))\\
& = f^*(\omega_1)_p(v_1, \dots, v_k) + f^*(\omega_2)_p(v_1,\dots,v_k)
\end{align*}
y
\begin{align*}
f^*(h \cdot \omega_1)_p(v_1, \dots, v_k) &= (h \cdot \omega_1)_{f(p)}(f_{\ast,p}(v_1),\dots,f_{\ast,p}(v_k))\\
& = h(f(p)) \cdot {\omega_1}_{f(p)}(f_{\ast,p}(v_1),\dots,f_{\ast,p}(v_k))\\
&= h(f(p)) \cdot f^*(\omega_1)_p(v_1,\dots,v_k),
\end{align*}
así que $f^*(\omega_1+\omega_2) = f^*(\omega_1) + f^*(\omega_2)$ y $f^*(h\cdot \omega_1) = h\circ f\cdot f^*(\omega_1)$.

Por último, sean $\omega \in \Omega^k(N)$ y $\eta \in \Omega^l(N)$ dos formas. Ahora dado un punto $p \in N$ y derivaciones $v_1, \dots, v_k,v_{k+1},\dots,v_{k+l} \in T_pN \subset TN$, al tensorizar obtenemos
\begin{align*}
(\omega \otimes \eta)_{f(p)} (d_pf(v_1), \dots d_pf(v_{k+l})) &= \omega_{f(p)} (d_pf(v_{\sigma(1)}), \dots d_pf(v_{\sigma(k)})) \cdot \eta_{f(p)}(d_pf(v_{\sigma(k+1)}), \dots, d_pf(v_{\sigma(k+l)}))\\
& = f^*(\omega_p) (v_{\sigma(1)}, \dots v_{\sigma(k)}) \cdot f^*(\eta)_p(v_{\sigma(k+1)}, \dots, v_{\sigma(k+l)})\\
& = (f^*(\omega) \otimes f^*(\eta))_p(v_{1},\dots, v_{k+l})\\
\end{align*}
así que 
\begin{align*}
f^*(\omega \wedge \eta)_p(v_1, \dots, v_{k+l}) &= (\omega \wedge \eta)_{f(p)}(d_pf(v_1), \dots, d_pf(v_{k+l}))\\
&= \frac{1}{k!l!}\sum_{\sigma \in \Ss_{k+l}}(-1)^\sigma \sigma \cdot (\omega \otimes \eta)_{f(p)} (d_pf(v_1), \dots d_pf(v_{k+l}))\\
&= \frac{1}{k!l!}\sum_{\sigma \in \Ss_{k+l}}(-1)^\sigma (\omega \otimes \eta)_{f(p)} (d_pf(v_{\sigma(1)}), \dots d_pf(v_{\sigma(k+l)}))\\
&= \frac{1}{k!l!}\sum_{\sigma \in \Ss_{k+l}}(-1)^\sigma (f^*(\omega) \otimes f^*(\eta))_p(v_{\sigma(1)},\dots, v_{\sigma(k+l)})\\
& = \frac{1}{k!l!}\sum_{\sigma \in \Ss_{k+l}}(-1)^\sigma \sigma \cdot (f^*(\omega) \otimes f^*(\eta))_p(v_{1},\dots, v_{k+l})\\
&= (f^*(\omega) \wedge f^*(\eta))_p (v_{1},\dots, v_{k+l}).
\end{align*}
\item[b)] Fijemos $i \in \nat{n}$. Si $s \in \nat{n}$, entonces para cada $p \in U$ y $g \in C^\infty(V)$ es
\begin{align*}
d_pf\left(\frac{\partial}{\partial x_s}\Big|_p\right)(g) = \frac{\partial gf}{\partial x_s}\Big|_p = (D_p(gf))_s = (D_{f(p)}gD_pf)_s = \sum_{j = 1}^n\frac{\partial g}{\partial x_j}\Big|_{f(p)} \cdot \frac{\partial f_j}{\partial x_s}\Big|_p = \sum_{j = 1}^n\frac{\partial f_j}{\partial x_s}\Big|_p\cdot \frac{\partial}{\partial x_j}\Big|_{f(p)}(g)
\end{align*}
de forma que $d_pf(\frac{\partial}{\partial x_s}|_p) = \sum_{j = 1}^n\frac{\partial f_j}{\partial x_s}|_p\cdot \frac{\partial}{\partial x_j}|_{f(p)}$. Por lo tanto,
\begin{align*}
f^*(\d x_i)_p\left(\frac{\partial}{\partial x_s}\Big|_p\right) &= d{x_i}_{f(p)}\left(d_pf\left(\frac{\partial}{\partial x_s}\Big|_p\right)\right) = d{x_i}_{f(p)}\left(\sum_{j = 1}^n\frac{\partial f_j}{\partial x_s}\Big|_p\cdot \frac{\partial}{\partial x_j}\Big|_{f(p)}\right)\\
& = \sum_{j = 1}^n\frac{\partial f_j}{\partial x_s}\Big|_p\cdot d{x_i}_{f(p)}\left(\frac{\partial}{\partial x_j}\Big|_{f(p)}\right) = \frac{\partial f_i}{\partial x_s}.
\end{align*}

Como a su vez 
\begin{align*}
\sum_{k=1}^n\frac{\partial f_i}{\partial x_k}(\d {x_k})_p\left(\frac{\partial}{\partial x_s}\Big|_{p}\right) = \sum_{k=1}^n\frac{\partial f_i}{\partial x_k}\delta_{ks} = \frac{\partial f_i}{\partial x_s}
\end{align*}
obtenemos que 
\begin{align*}
f^*(\d x_i)_p \sum_{k=1}^n\frac{\partial f_i}{\partial x_k}(\d {x_k})_{p}
\end{align*}
pues ambos lados de la igualdad coinciden en la base de los \textit{ganchos} $\{\frac{\partial}{\partial x_s}|_p\}_{1 \leq s \leq n}$. Como esto es cierto para todo punto $p$, se tiene que $f^*(\d x_i) = \sum_{k=1}^n\frac{\partial f_i}{\partial x_k}\d {x_k}$.

Finalmente, usando $\paint{(a)}$ vemos que
\begin{align*}
f^*(g\cdot\d x_1\wedge\cdots\wedge\d x_n) &= g \circ f \cdot f^*(\d x_1)\wedge\dots\wedge f^*(\d x_n)
\end{align*}
así que para terminar el inciso basta ver que
\begin{align*}
f^*(\d x_1)\wedge\dots\wedge f^*(\d x_n) = \det(\frac{\partial f_i}{\partial x_j})_{ij} \cdot  \d x_1 \wedge\dots\wedge\d x_n.
\end{align*}

En efecto,
\begin{align*}
f^*(\d x_1)\wedge\dots\wedge f^*(\d x_n) &= \sum_{i_1=1}^n\frac{\partial f_1}{\partial x_{i_1}}\d {x_{i_1}} \wedge \dots \wedge \sum_{i_n=1}^n\frac{\partial f_n}{\partial x_{i_n}}\d {x_{i_n}}\\
&= \sum_{i_1, \dots, i_n}\prod_{j=1}^n\frac{\partial f_j}{\partial x_{i_j}} \d x_{i_1} \wedge\dots\wedge \d x_{i_n}\\
&= \sum_{i_1 < \dots < i_n} \prod_{j=1}^n\frac{\partial f_j}{\partial x_{i_j}} \d x_{i_1} \wedge\dots\wedge \d x_{i_n}\\
&= \sum_{\sigma \in \Ss_n} \prod_{j=1}^n\frac{\partial f_j}{\partial x_{\sigma(j)}} \d x_{\sigma(1)} \wedge\dots\wedge \d x_{\sigma(n)}\\
&= \sum_{\sigma \in \Ss_n} (-1)^\sigma \prod_{j=1}^n\frac{\partial f_j}{\partial x_{\sigma(j)}} \d x_1 \wedge\dots\wedge \d x_n\\
&= \left(\sum_{\sigma \in \Ss_n} (-1)^\sigma \prod_{j=1}^n\frac{\partial f_j}{\partial x_{\sigma(j)}}\right) \d x_1 \wedge\dots\wedge \d x_n\\
&= \det\left(\frac{\partial f_i}{\partial x_j}\right)_{ij} \cdot  \d x_1 \wedge\dots\wedge\d x_n.
\end{align*}
\end{itemize}
\end{proof}

\begin{exercise}{11 (d)} Calcule la cohomología de un producto cartesiano de dos esferas.
\begin{proof} Sea $n \in \N$. Por la fórmula de Künneth, sabemos que
\begin{align}
H^q(\Ss^n \times \Ss^n) = \bigoplus_{r+s = q} H^r(\Ss^n) \otimes H^s(\Ss^n)
\end{align}
para todo $q \geq 0$. Sabemos además que la cohomología de la $n$-esfera es $\R$ en grados $0$ y $n$, y nula en los demás. Por lo tanto, para que $H^r(\Ss^n) \otimes H^s(\Ss^n)$ sea no nulo es condición necesaria que $r$ y $s$ pertenezcan a $\{0,n\}$. En particular debe ser $r+s \in \{0,n,2n\}$. 

Como $\Ss^n \times  \Ss^n$ es arcoconexo, ya sabemos que $H^0(\Ss^n \times \Ss^n) \simeq \R$. En los otros casos, de $\paint{(1)}$ tenemos que
\begin{align*}
H^n(\Ss^n \times \Ss^n) &= H^n(\Ss^n) \otimes H^0(\Ss^n) \oplus H^0(\Ss^n) \otimes H^n(\Ss^n) = \R \otimes \R \oplus \R \otimes \R = \R \oplus \R
\end{align*}
y
\begin{align*}
H^{2n}(\Ss^n \times \Ss^n) = H^n(\Ss^n) \otimes H^n(\Ss^n) = \R \otimes \R = \R.
\end{align*}
Por lo tanto, es
\begin{align*}
H^q(\Ss^n \times \Ss^n) = \begin{cases}
\R \quad &\text{si $q = 0$ o $q = 2n$}\\
\R^2 \quad &\text{si $q = n$}\\
0 \quad &\text{en caso contrario}
\end{cases}
\end{align*}
\end{proof}
\end{exercise}


\end{document}
