\documentclass[11pt]{article}
\usepackage[margin=1in]{geometry} 
\usepackage{amsmath,amsthm,amssymb,amsfonts}
\usepackage[utf8]{inputenc}
\usepackage[T1]{fontenc}
\usepackage{microtype}
\usepackage{mathpazo}
\usepackage{euler}
\usepackage{xcolor}
\usepackage{tikz}
\usepackage{tikz-cd}
\usetikzlibrary{arrows}
\usetikzlibrary{matrix}
\usepackage{fancyhdr}
\pagestyle{fancy}
\usepackage{enumitem}

\newcommand{\N}{\mathbb{N}}
\newcommand{\Z}{\mathbb{Z}}
\newcommand{\Q}{\mathbb{Q}}
\newcommand{\R}{\mathbb{R}}
\newcommand{\C}{\mathbb{C}}
\newcommand{\Ss}{\mathbb{S}}
\newcommand{\M}[2]{\mathsf{M}_{#1}#2}
\newcommand{\X}{\mathfrak{X}}
\renewcommand{\div}{\operatorname{div}}
\newcommand{\grad}{\operatorname{grad}}
\newcommand{\im}{\operatorname{im}}
\newcommand{\eps}{\varepsilon}
\newcommand{\dpart}[2]{\frac{\partial#1}{\partial#2}}
\newcommand{\nat}[1]{[\![#1]\!]}
\newcommand{\natzero}[1]{\nat{#1}_0}
\newcommand{\adj}[1]{\operatorname{adj}(#1)}
\newcommand{\ip}[1]{\langle #1 \rangle}
\newcommand{\ol}{\overline}
\newcommand{\hook}[3]{\frac{\partial}{\partial x_{#1}}\Big\rvert_{#2}^{#3}}
\usepackage{rotating}
\newcommand*{\isoarrow}[1]{\arrow[#1,"\rotatebox{90}{\LARGE{\(\sim\)}}"
]}
\renewcommand{\d}{\operatorname{d}}
\definecolor{color}{RGB}{0, 17, 102}
\newcommand{\paint}[1]{\color{color}{#1}}
\newcommand{\tpaing}[1]{\paint{\text{#1}}}
\newcommand{\paintline}{\begin{center}
$\paint{
\rule{400pt}{0.5pt}
}$
\vspace{10pt}
\end{center}}

\renewcommand*{\proofname}{\paint{Demostraci\'on}}
\newenvironment{theorem}[2][Teorema]{\begin{trivlist}
\item[\hskip \labelsep \paint{{\bfseries #1}}\hskip \labelsep {\bfseries #2.}]}{\end{trivlist}}
\newenvironment{lemma}[2][Lema]{\begin{trivlist}
\item[\hskip \labelsep \paint{{\bfseries #1}}\hskip \labelsep {\bfseries #2.}]}{\end{trivlist}}
\newenvironment{exercise}[2][Ejercicio]{\begin{trivlist}
\item[\hskip \labelsep \paint{{\bfseries #1}}\hskip \labelsep {\bfseries #2.}]}{\end{trivlist}}
\newenvironment{obs}[2][Observaci\'on]{\begin{trivlist}
\item[\hskip \labelsep \paint{{\bfseries #1.}}]}{\end{trivlist}}
\newenvironment{reflection}[2][Resoluci\']{\begin{trivlist}
\item[\hskip \labelsep {\bfseries #1}\hskip \labelsep {\bfseries #2.}]}{\end{trivlist}}
\newenvironment{proposition}[2][Proposici\'on]{\begin{trivlist}
\item[\hskip \labelsep \paint{{\bfseries #1}}\hskip \labelsep {\bfseries #2.}]}{\end{trivlist}}
\newenvironment{corollary}[2][Corolario]{\begin{trivlist}
\item[\hskip \labelsep {\bfseries #1}\hskip \labelsep {\bfseries #2.}]}{\end{trivlist}}

%-----------------------

\title{
\LARGE{\paint{Geometr\'ia Diferencial}}
\\
\vspace{1pt}
\small\paint{Primer Cuatrimestre -- 2019}
\\
\vspace{0.5pt}
\large{\paint{Segundo Parcial}}
}
\author{\paint{Guido Arnone}}
\date{}
\lhead{Guido Arnone}
\rhead{Segundo Parcial}

\begin{document}

\maketitle

\begin{center}
$\paint{
\rule{400pt}{0.5pt}
}$
\vspace{35pt}
\end{center}

\begin{exercise}{1} Sea $M$ una variedad riemanniana, sea $\nabla$ la conexión de Levi-Civita de $M$ y sea $f : M \to \R$ una función diferenciable.
\begin{itemize}[listparindent = \parindent]
\item[(a)] Muestre que existe un campo vectorial diferenciable $\grad(f) \in \X(M)$ y uno solo con la propiedad de que para cada campo $Y \in \X(M)$ se tiene que
\begin{align*}
\ip{\grad(f), Y} = df(Y).
\end{align*}
Encuentre una expresión en coordenadas para el campo $\grad(f)$.
\item[(b)] La función 
\begin{align*}
X \in \X(M) \mapsto \nabla_X\grad(f) \in \X(M)
\end{align*}
es autoadjunta: cada vez que $X$ e $Y$ son elementos de $\X(M)$ se tiene que
\begin{align*}
\ip{\nabla_X\grad(f),Y} = \ip{X,\nabla_Y\grad(f)}.
\end{align*}
\item[(c)] Muestre que si el campo $\grad(f)$ tiene norma constante, entonces para todo $X \in \X(M)$ se tiene que $\ip{\nabla_{\grad(f)}\grad(f),X} = 0$. Deduzca de esto que bajo esa condición las curvas integrales de $\grad(f)$ son geodésicas.
\end{itemize}
\end{exercise}
\begin{proof}
content...
\end{proof}

\paintline

\begin{exercise}{2} Sean $M$ y $N$ dos variedades compactas, conexas, orientadas y de la misma dimensión $n$ y sea $f : M \to N$ una función diferenciable.
\begin{itemize}
\item[(a)] Muestre que hay un número real $\lambda \in \R$ tal que para toda forma $\omega \in \Omega^n(N)$ se tiene 
\begin{align*}
\int_Mf^*(\omega) = \lambda\int_N \omega.
\end{align*}
Lo llamamos \textit{grado} de $f$ y lo escribimos $\deg(f)$.
\item[(b)] Supongamos que $q \in N$ es valor regular de $f$, de manera que, en particular, el conjunto $f^{-1}(q)$ es finito. Si $p \in f^{-1}(q)$ la diferencial es entonces un isomorfismo de espacios vectoriales y podemos considerar el número
\begin{align*}
sgn_f(p) = \begin{cases}
+1&\text{ si $d_pf$ preserva la orientación}\\
-1&\text{si la invierte}
\end{cases}
\end{align*}
ya que esas son las dos únicas posibilidades.

Muestre que
\begin{align*}
\deg(f) = \sum_{p \in f^{-1}(q)}sgn_f(p).
\end{align*}
\end{itemize}
\end{exercise}
\begin{proof}
content...
\end{proof}

\paintline

\begin{exercise}{3} Muestre que cuando $n \geq 2$ el grado de toda función diferenciable $f : \Ss^n \to T^n$ de la $n$-esfera al $n$-toro es nulo.
\end{exercise}
\begin{proof}
content...
\end{proof}

\paintline

\begin{theorem}{1 (dualidad de Poincaré)} Sea $M$ una variedad conexa y orientable de dimensión $n$. Entonces, para cada $k \in \nat{n}$ se tiene que $H^k(M) \simeq H^{n-k}_c(M)^*$.
\end{theorem}

\begin{exercise}{4} Sea $M$ una variedad compacta, orientable, conexa de dimensión $n$. Sabemos que la cohomología de De Rham de $M$ tiene entonces dimensión total finita, y podemos en consecuencia considerar el entero
\begin{align*}
\chi(M) = \sum_{i=0}^{n}(-1)^i \dim H^i(M)
\end{align*}
al que llamamos la \textit{característica de Euler} de $M$.
\begin{itemize}
\item[(a)] Si la dimensión $n$ de $M$ es impar, entonces $\chi(M) = 0$.
\item[(b)] Si la dimensión $n$ de $M$ es par y la de $H^{n/2}(M)$ es par, entonces $\chi(M)$ es un entero par.
\end{itemize}
\end{exercise}
\begin{proof} Notemos que como la variedad $M$ es compacta, su cohomología a soporte compacto coincide con la cohomología de de Rham «a secas». Además, dado que la cohomología de una variedad compacta es finitamente generada, se tiene que
\begin{align*}
H^{n-i}_c(M)^* \simeq H^{n-i}_c(M) = H^{n-i}(M)
\end{align*}
para cada $i \in \nat{n}$.

Por el $\tpaing{Teorema $1$}$, tenemos entonces que $H^i(M) \simeq H^{n-i}(M)$ para todo $i \in \nat{n}$. En particular, si notamos $\beta_i := \dim H^i(M)$ a la dimensión del $i$-ésimo grupo de cohomología, debe ser $\beta_i = \beta_{n-i}$.
Ahora,
\begin{itemize}
\item[(a)] Si $n$ es impar, entonces
\begin{align*}
2\chi(M) &= \sum_{i=0}^{n}(-1)^i\beta_i + \sum_{i=0}^{n}(-1)^i\beta_i =
\sum_{i=0}^{n}(-1)^i\beta_i + \sum_{i=0}^{n}(-1)^{n-i}\beta_{n-i}\\
&= \sum_{i=0}^{n}(-1)^i\beta_i + (-1)^n\sum_{i=0}^{n}(-1)^{-i}\beta_{n-i}\\
&= \sum_{i=0}^{n}(-1)^i\beta_i + (-1)^n\sum_{i=0}^{n}(-1)^{i}\beta_{i}\\
&= \chi(M)(1 + (-1)^n) = \chi(M) \cdot  0 = 0.
\end{align*}
Por lo tanto, es $\chi(M) = 0$.
\item[(b)] De una forma similar, si $n$ y es par y $\beta_{n/2}$ también, entonces
\begin{align*}
\chi(M) &= \sum_{i=0}^{n}(-1)^i\beta_i = \sum_{i=0}^{n/2-1}(-1)^i\beta_i + \beta_{n/2} + \sum_{i=n/2+1}^{n}(-1)^i\beta_i\\
&= \sum_{i=0}^{n/2-1}(-1)^i\beta_i + \beta_{n/2} + \sum_{i=0}^{n/2-1}(-1)^{n-i}\beta_{n-i}\\
&= 2\sum_{i=0}^{n/2-1}(-1)^i\beta_i + \beta_{n/2} \equiv \beta_{n/2} \equiv 0 \pmod{2}.
\end{align*}
En consecuencia, la característica de Euler de $M$ es par.
\end{itemize}
\end{proof}


\paintline

\begin{exercise}{5} Sea $G$ un grupo de Lie de dimensión $n$, sea $\mathfrak{g} = T_eG$ su álgebra de Lie y fijemos un producto interno $g_e : \mathfrak{g} \times \mathfrak{g} \to \R$ sobre $\mathfrak{g}$.
\begin{itemize}
\item[(a)] Hay una única métrica riemanniana $g$ sobre $G$ que es invariante a izquierda y cuyo valor en $e \in G$ es $g_e$.
\item[(b)] Sea $\mathscr{B} = \{v_1, \dots, v_n\}$ una base de $\mathfrak{g}$ y sean $X_1, \dots, X_n$ los campos tangentes a $G$ invariantes a izquierda que extienden a los elementos de $\mathscr{B}$. Muestre que para cada $i,j$, es \textit{constante} la función $g_{i,j} = g(X_i,X_j)$.

Sabemos (porque el corchete de Lie de campos invariantes a izquierda es él mismo invariante a izquierda) que existen constantes $c_{i,j}^k$ tales que
\begin{align*}
[X_i,X_j] = \sum_{k}c_{i,j}^kX_k.
\end{align*}
Calcule en términos de los escalares $c_{i,j}^k$ y $g_{i,j}$ los símbolos de Christoffel $\Gamma_{ij}^k$ de la conexión de Levi-Civita de G con respecto a los campos $X_1, \dots, X_n$, de manera que se tenga
\begin{align*}
\nabla_{X_i}X_j = \sum_k\Gamma_{ij}^kX_k.
\end{align*}
\item[(c)] Sea $G = \R_{>0 } \times \R$ el grupo de Lie con producto dado por
\begin{align*}
(a,b) \cdot (c,d) = (ac,ad+b)
\end{align*}
para cada $(a,b),(c,d) \in G$, de manera que $G$ es isomorfo de la forma evidente al grupo de matrices 
\begin{align*}
\left\{\begin{pmatrix}
a & b \\
0 & 1
\end{pmatrix} : a >0, b \in \R\right\}.
\end{align*}
El elemento neutro $G$ es $e = (1,0)$ y su álgebra de Lie $\mathfrak{g} = T_eG$ se identifica de manera natural (porque $G$ es un abierto de $\R^2$) con $\R^2$. Dotemos a $G$ de su única métrica invariante a izquierda que en $T_eG$ restringe al producto interno usual de $\R^2$. Encuentre todas las geodésicas que pasan por $e$ que pueda.

Calcule (las componentes en una carta del) tensor de curvatura $R(X,Y)Z$ sobre $G$ y la \textit{curvatura escalar}
\begin{align*}
K(p) = \frac{1}{n(n-1)} \sum_{1 \leq i,j \leq n}g(R(z_i,z_j)z_i,z_j)
\end{align*}
para cada $p \in G$, con $\{z_1, \dots, z_n\}$ una base ortonormal de $T_pG$.
\end{itemize}
\end{exercise}
\begin{proof}
content...
\end{proof}

\paintline

\begin{exercise}{6} Sea $M$ una variedad compacta y orientable de dimensión $4k$.
\begin{itemize}
\item[(a)] Muestre que hay una función bilineal no degenerada y simétrica
\begin{align*}
\sigma : H^{2k}(M) \times H^{2k}(M) \to \R
\end{align*}
tal que si $\omega$ y $\eta$ son elementos cerrados de $\Omega^{2k}(M)$ entonces 
\begin{align*}
\beta([\omega],[\eta]) = \int_M \omega \wedge \eta.
\end{align*}
Llamamos la signatura de la forma bilineal $\sigma$ la signatura de $M$.
\item[(b)] Determine la signatura de $\Ss^4$, de $\Ss^2 \times \Ss^2$, del toro $T^4$, del espacio proyectivo $P_{\C}^2$ y el producto $P_{\C}^2 \times P_{\C}^2$.
\end{itemize}
\end{exercise}
\begin{proof}
content...
\end{proof}

\paintline

\begin{exercise}{7} Sea $M \subseteq \R^3$ una superficie orientada y sin borde dotada de su métrica riemanniana inducida por la de $\R^3$ y supongamos que hay un campo vectorial $Z \in \X(M)$ sobre $M$ que no se anula en ningún punto.
\begin{itemize}
\item[(a)] Muestre que existe una única forma de elegir campos $X,Y \in \X(M)$ tales que para cada $p \in M$ se tiene que $(X_p,Y_p)$ es una base ortonormal positiva de $T_pM$ y $Z_p = \|Z_p\|X_p$. 
\item[(b)] Hay $1$-formas $\alpha,\beta\in \Omega^1(M)$ tales que $\alpha(X) = \beta(Y) = 1$ y $\alpha(Y) = \beta(X) = 0$. Más aún, existe una forma $\eta \in \Omega^1(M)$ tal que
\begin{align*}
d\alpha = \eta \wedge \beta, \quad d\beta = -\eta \wedge \alpha.
\end{align*}
La forma $\sigma  = \alpha \wedge \beta$ no depende de la elección de $Z$, es una forma de volumen sobre $M$ que determina su orientación y es, de hecho, la forma de volumen riemanniano sobre $M$.
\item[(c)] Existe una función diferenciable $K : M \to \R$ tal que
\begin{align*}
d\eta = -K \cdot \sigma
\end{align*}
y esta función no depende de la elección del campo $Z$.
\item[(d)] Si $M$ es compacta, entonces $\int_M K \cdot \sigma = 0$.
\item[(e)] Usando los resultados anteriores, muestre que no hay sobre $\Ss^2$ un campo vectorial tangente que no se anula en ningún punto.
\end{itemize}
\end{exercise}
\begin{proof}
content...
\end{proof}

\end{document}
