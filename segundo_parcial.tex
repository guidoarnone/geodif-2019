\documentclass[11pt]{article}
\usepackage[margin=1in]{geometry} 
\usepackage{amsmath,amsthm,amssymb,amsfonts}
\usepackage[utf8]{inputenc}
\usepackage[T1]{fontenc}
\usepackage{microtype}
\usepackage{mathpazo}
\usepackage{euler}
\usepackage{xcolor}
\usepackage{tikz}
\usepackage{tikz-cd}
\usetikzlibrary{arrows}
\usetikzlibrary{matrix}
\usepackage{fancyhdr}
\pagestyle{fancy}
\usepackage{enumitem}
\usepackage{tcolorbox}
\tcbuselibrary{theorems}

\newcommand{\N}{\mathbb{N}}
\newcommand{\Z}{\mathbb{Z}}
\newcommand{\Q}{\mathbb{Q}}
\newcommand{\R}{\mathbb{R}}
\newcommand{\C}{\mathbb{C}}
\newcommand{\Ss}{\mathbb{S}}
\newcommand{\T}{\mathbb{T}}
\newcommand{\M}[2]{\mathsf{M}_{#1}#2}
\newcommand{\X}{\mathfrak{X}}
\renewcommand{\div}{\operatorname{div}}
\newcommand{\grad}{\operatorname{grad}}
\newcommand{\im}{\operatorname{im}}
\newcommand{\sop}{\operatorname{sop}}
\newcommand{\eps}{\varepsilon}
\newcommand{\dpart}[2]{\frac{\partial#1}{\partial#2}}
\newcommand{\nat}[1]{[\![#1]\!]}
\newcommand{\natzero}[1]{\nat{#1}_0}
\newcommand{\adj}[1]{\operatorname{adj}(#1)}
\newcommand{\ip}[1]{\langle #1 \rangle}
\newcommand{\ol}{\overline}
\newcommand{\hook}[3]{\frac{\partial}{\partial x_{#1}}\Big\rvert_{#2}^{#3}}
\usepackage{rotating}
\newcommand*{\isoarrow}[1]{\arrow[#1,"\rotatebox{90}{\LARGE{\(\sim\)}}"
]}
\newcommand{\gancho}[1]{\frac{\partial}{\partial \varphi^{#1}}}
\renewcommand{\d}{\operatorname{d}}
\definecolor{color}{RGB}{0, 17, 102}
\newcommand{\paint}[1]{\color{color}{#1}}
\newcommand{\tpaing}[1]{\paint{\text{#1}}}
\newcommand{\paintline}{\begin{center}
$\paint{
\rule{400pt}{0.5pt}
}$
\vspace{10pt}
\end{center}}

\renewcommand*{\proofname}{\paint{Demostraci\'on}}
\newenvironment{theorem}[2][Teorema]{\begin{trivlist}
\item[\hskip \labelsep \paint{{\bfseries #1}}\hskip \labelsep {\bfseries #2.}]}{\end{trivlist}}
\newenvironment{lemma}[2][Lema]{\begin{trivlist}
\item[\hskip \labelsep \paint{{\bfseries #1}}\hskip \labelsep {\bfseries #2.}]}{\end{trivlist}}
\newenvironment{exercise}[2][Ejercicio]{\begin{trivlist}
\item[\hskip \labelsep \paint{{\bfseries #1}}\hskip \labelsep {\bfseries #2.}]}{\end{trivlist}}
\newenvironment{obs}[2][Observaci\'on]{\begin{trivlist}
\item[\hskip \labelsep \paint{{\bfseries #1.}}]}{\end{trivlist}}
\newenvironment{reflection}[2][Resoluci\']{\begin{trivlist}
\item[\hskip \labelsep {\bfseries #1}\hskip \labelsep {\bfseries #2.}]}{\end{trivlist}}
\newenvironment{proposition}[2][Proposici\'on]{\begin{trivlist}
\item[\hskip \labelsep \paint{{\bfseries #1}}\hskip \labelsep {\bfseries #2.}]}{\end{trivlist}}
\newenvironment{corollary}[2][Corolario]{\begin{trivlist}
\item[\hskip \labelsep {\bfseries #1}\hskip \labelsep {\bfseries #2.}]}{\end{trivlist}}

%-----------------------

\title{
\LARGE{\paint{Geometr\'ia Diferencial}}
\\
\vspace{1pt}
\small\paint{Primer Cuatrimestre -- 2019}
\\
\vspace{0.5pt}
\large{\paint{Segundo Parcial}}
}
\author{\paint{Guido Arnone}}
\date{}
\lhead{Guido Arnone}
\rhead{Segundo Parcial}

\begin{document}

\maketitle
\begin{center}
\paint{\large{Sobre la Resolución}}
\end{center}

Con la intención de hacer más legible el examen, algunos argumentos están escritos en forma de lemas que preceden a cada ejercicio. También incluyo (sin demostración) los resultados vistos en clase que utilicé.
\begin{center}
$\paint{
\rule{400pt}{0.5pt}
}$
\vspace{15pt}
\end{center}

\begin{exercise}{1} Sea $M$ una variedad riemanniana, sea $\nabla$ la conexión de Levi-Civita de $M$ y sea $f : M \to \R$ una función diferenciable.
\begin{itemize}[listparindent = \parindent]
\item[(a)] Muestre que existe un campo vectorial diferenciable $\grad(f) \in \X(M)$ y uno solo con la propiedad de que para cada campo $Y \in \X(M)$ se tiene que
\begin{align*}
\ip{\grad(f), Y} = df(Y).
\end{align*}@A
Encuentre una expresión en coordenadas para el campo $\grad(f)$.
\item[(b)] La función 
\begin{align*}
X \in \X(M) \mapsto \nabla_X\grad(f) \in \X(M)
\end{align*}
es autoadjunta: cada vez que $X$ e $Y$ son elementos de $\X(M)$ se tiene que
\begin{align*}
\ip{\nabla_X\grad(f),Y} = \ip{X,\nabla_Y\grad(f)}.
\end{align*}
\item[(c)] Muestre que si el campo $\grad(f)$ tiene norma constante, entonces para todo $X \in \X(M)$ se tiene que $\ip{\nabla_{\grad(f)}\grad(f),X} = 0$. Deduzca de esto que bajo esa condición las curvas integrales de $\grad(f)$ son geodésicas.
\end{itemize}
\end{exercise}
\begin{proof} Hago cada inciso por separado.
\begin{itemize}[listparindent = \parindent]
\item[(a)] Notemos en primer lugar que la función $f$ induce una $1$-forma $df$ que en cada punto $p \in M$ vale $d_pf : v \in T_pM \mapsto v(f) \in \R$, el diferencial de $f$ bajo la identificación $T_p\R \simeq \R$. 

Fijemos ahora $p \in M$. Como $d_pf \in (T_pM)^*$ es un elemento del espacio dual de $T_pM$, y este último es un $\R$-espacio vectorial de dimensión finita equipado con un producto interno (inducido por la métrica de $M$), el teorema de representación de Riesz nos asegura que existe un único vector tangente $v_p \in T_pM$ tal que 
\begin{align}
\ip{v_p,w} = d_pf(w) = w(f) \quad (\forall w \in T_pM).
\end{align}
Si definimos $\grad_p(f) := v_p$ para cada $p \in M$, reescribiendo la anterior igualdad es
\begin{align*}
\ip{\grad(f)_p,w} = d_pf(w) \quad (\forall w \in T_pM).
\end{align*}
y éste es el único campo con tal propiedad. En particular, si $Y \in \X(M)$ entonces para cada $p \in M$ es
\begin{align*}
(\ip{\grad(f),Y})_p = \ip{\grad(f)_p,Y_p} = d_pf(Y_p) = (df(Y))_p
\end{align*}
y por lo tanto $\ip{\grad(f),Y} \equiv df(Y)$. 

Para ver la unicidad, recordemos que para cada $p \in M$ y $v \in  T_pM$ existe $Y^v \in \X(M)$ con $Y^v_p = v$. Efectivamente, podemos tomar una carta $(U, \varphi)$ con $U \ni p$ de forma que existan coeficientes $a_1, \dots, a_n$ tales que $v = \sum_{1 \leq i \leq n}a_i \gancho{i}|_p$ y luego tomar el campo
\begin{align*}
Y^v = h \cdot \sum_{i = 1}^na_i \gancho{i}
\end{align*}
con $h \in C^\infty(M)$ una función \textit{bump} (tal que valga $1$ en un entorno abierto $V \subset U$  de $p$ y $0$ en un abierto $W \supset U^c$). 

A partir de esto, podemos concluir que cualquier otro campo $Z \in \X(M)$ que cumpla $\ip{Z,Y} \equiv df(Y)$ para todo $Y \in \X(M)$ deberá satisfacer
\begin{align*}
\ip{Z_p,v} = \ip{Z_p,Y^v_p} = d_pf(Y_p^v) = v(f)
\end{align*}
para todo $p \in M$ y $v \in T_pM$. La unicidad de $\paint{(1)}$ nos dice entonces que $Z_p = v_p = \grad_p(f)$ en todo punto $p \in M$.

Para terminar, veamos una expresión de $\grad_p(f)$ en coordenadas. De aquí se tendrá que el gradiente depende localmente de funciones suaves, y es por lo tanto diferenciable.

Una vez más, fijamos $p \in M$ y consideramos $(\varphi,U)$ una carta de $M$ tal que $U \ni p$. Como los \textit{ganchos} $\{\gancho{1}|_p,\dots,\gancho{n}|_p\}$ son una base de $T_pM$, sabemos que existen únicos coeficientes $c_1, \dots, c_n \in \R$ tales que
\begin{align*}
v_p = \sum_{j = 1}^n c_j \cdot  \gancho{j}\Big|_p.
\end{align*} 
Si ahora tomamos el producto interno de $v_p$ con $\gancho{i}|_p$, es
\begin{align*}
\frac{\partial f}{\partial \varphi^i}\Big|_p = \gancho{i}\Big|_p(f) = \left\ip{v_p,\gancho{i}\Big|_p\right} = \sum_{j = 1}^n c_j \cdot \left\ip{\gancho{j}\Big|_p,\gancho{i}\Big|_p\right} = \sum_{j = 1}^n c_j \cdot (g_p)_{ji}
\end{align*}
para cada $j \in \nat{n}$, y notando $c = (c_1, \dots, c_n)$ esto es equivalente a 
\begin{align*}
c \cdot g_p = \left(\frac{\partial f}{\partial \varphi^1}\Big|_p, \dots, \frac{\partial f}{\partial \varphi^n}\Big|_p\right).
\end{align*}
Por lo tanto debe ser $c = (\frac{\partial f}{\partial \varphi^1}\big|_p, \dots, \frac{\partial f}{\partial \varphi^n}\big|_p)(g_p)^{-1}$ y
\begin{align*}
c_j = \sum_{i=1}^n \frac{\partial f}{\partial \varphi^i}\Big|_p (g_p)^{ij}.
\end{align*}
Volviendo a la expresión original, obtenemos finalmente
\begin{align*}
\grad_p(f) = v_p = \sum_{j = 1}^n \left(\sum_{i=1}^n \frac{\partial f}{\partial \varphi^i}\Big|_p (g_p)^{ij}\right) \cdot \gancho{j}\Big|_p = \sum_{i,j}(g_p)^{ij} \frac{\partial f}{\partial \varphi^i}\Big|_p \cdot \gancho{j}\Big|_p.
\end{align*}

Esto prueba que para todo $p \in U$ se tiene (contrayendo indices) que
\begin{align*}
\grad(f) = g^{ij}\frac{\partial f}{\partial \varphi^i}\gancho{j}.
\end{align*}
Como afirmamos, esto prueba además que $\grad(f)$ es diferenciable, ya que para cada punto tenemos un abierto donde este es un campo suave.
\item[(b)] Como $\nabla$ es la conexión de Levi-Civita, sabemos (por construcción) que esta es compactible con la métrica y libre de torsion. Concretamente, si $X,Y,Z \in \X(M)$ entonces
\begin{align}
X\ip{Y,Z} &= \ip{\nabla_XY,Z} + \ip{Y,\nabla_XZ}
\end{align}
y
\begin{align}
\nabla_XY - \nabla_YX &= [X,Y].
\end{align}
Sean ahora $X,Y \in \X(M)$ dos campos en $M$. Por $\paint{(2)}$ y $\paint{(3)}$ sabemos que
\begin{align*}
\ip{\nabla_X\grad(f),Y} &= X\ip{\grad(f),Y}-\ip{\grad(f),\nabla_XY}\\
&= X\ip{\grad(f),Y} - \ip{\grad(f),[X,Y]+\nabla_YX}\\
&= X\ip{\grad(f),Y} - \ip{\grad(f),\nabla_YX} - \ip{\grad(f),[X,Y]}\\
&= X\ip{\grad(f),Y} - (Y\ip{\grad(f),X}-\ip{\nabla_Y\grad(f),X}) - \ip{\grad(f),[X,Y]}\\
&= \ip{\nabla_Y\grad(f),X} + X\ip{\grad(f),Y} - Y\ip{\grad(f),X} - \ip{\grad(f),[X,Y]}.
\end{align*} 

En consecuencia, se tiene que $\ip{\nabla_X\grad(f),Y} = \ip{X,\nabla_Y\grad(f)}$ si y solo si
\begin{align*}
X\ip{\grad(f),Y} - Y\ip{\grad(f),X} = \ip{\grad(f),[X,Y]}
\end{align*}
o lo que es lo mismo,
\begin{align*}
Xdf(Y) - Ydf(X) = df([X,Y]).
\end{align*}

Para terminar, observemos que esto ocurre siempre, pues
\begin{align*}
Xdf(Y) - Ydf(X) = XY(f) - YX(f) = (XY-YX)(f) = [X,Y](f) = df([X,Y]).
\end{align*}
\item[(c)] Supogamos que $\grad(f)$ tiene norma constante. Entonces $\|\grad(f)\|^2 = \ip{\grad(f),\grad(f)}$ debe valer constantemente $c$ para cierto $c \in \R$. Si ahora tomamos un campo $X \in \X(M)$, para todo $p \in M$ debe ser
\begin{align*}
(X\|\grad(f)\|)_p = X_p(c) = 0,
\end{align*}
y por lo tanto $X \ip{\grad(f),\grad(f)} \equiv 0$. 

Usando la compatibilidad con la métrica de la conexión de Levi-Civita, de la anterior igualdad se desprende que 
\begin{align*}
0 &= X \ip{\grad(f),\grad(f)} = \ip{\nabla_X\grad(f),\grad(f)} + \ip{\grad(f),\nabla_X\grad(f)}\\
&= 2\ip{\nabla_X\grad(f),\grad(f)}
\end{align*}
y por $\paint{(b)}$ es
\begin{align*}
\ip{\nabla_{\grad(f)}\grad(f),X} = \ip{\nabla_X\grad(f),\grad(f)} = 0.
\end{align*}

Por último, si $\gamma : I \to M$ es una curva integral de $\grad(f)$, entonces es $\nabla_{\dot{\gamma}}\dot{\gamma} \equiv 0$  pues
\begin{align*}
\|\nabla_{\dot{\gamma}(t)}\dot{\gamma}(t)\|^2 &= \ip{\nabla_{\dot{\gamma}(t)}\dot{\gamma}(t),\nabla_{\dot{\gamma}(t)}\dot{\gamma}(t)}\\
& = \ip{\nabla_{\grad(f)_{\gamma(t)}}\grad(f)_{\gamma(t)},\nabla_{\grad(f)_{\gamma(t)}}\grad(f)_{\gamma(t)}}\\
& = (\ip{\nabla_{\grad(f)}\grad(f),\nabla_{\grad(f)}\grad(f)})_{\gamma(t)} = 0
\end{align*}
para todo $t \in I$. Vemos así que las curvas integrales de $\grad(f)$ resultan geodésicas.
\end{itemize}
\end{proof}

\paintline

\begin{tcolorbox}
\begin{theorem}{1} Sea $M$ una variedad orientada, conexa y compacta de dimensión $n$. Entonces es $H^n(M) \simeq \R$ via el isomorfismo
\begin{align*}
[\omega] \in H^n&(M) \longmapsto \int_M \omega \in \R.
\end{align*}
\end{theorem}
\end{tcolorbox}

\begin{tcolorbox}
\begin{theorem}{2} Sean $M$ y $N$ variedades orientadas compactas y conexas de dimensión $n$. Si $f : M \to N$ es un difeomorfismo que preserva (resp. invierte) la orientación y $\omega \in \Omega^n(N)$ es una $n$-forma, entonces $\int_M f^*(\omega) = \int_N \omega$ (resp. $-\int_N \omega$). 
\end{theorem}
\end{tcolorbox}

\begin{lemma}{3} Sean $M$ y $N$ dos variedades y $f : M \to N$ una función suave. Si $q \in N$ es un valor regular y $f^{-1}(q) = \{p_1, \dots, p_k\}$, existe un entorno abierto $W \subset N$ de $q$ y abiertos conexos disjuntos $U_1, \dots, U_k$ tales que:
\begin{itemize}
\item[(i)] para cada $i \in \nat{k}$ es $U_i \ni p_i$,
\item[(ii)] la preimagen de $W$ por $f$ es $f^{-1}(W) = \bigsqcup_{i=1}^k U_i$, y
\item[(iii)] para cada $i \in \nat{k}$ la (co)restricción $f|_{U_i} : U_i \to W$ es un difeomorfismo.
\end{itemize}
\end{lemma}
\begin{proof} En primer lugar, como es $\dim  M = \dim N$ sabemos que en cada punto $p_i$ el diferencial de $f$ es un isomorfimo. Por el teorema de la función inversa, existen entonces abiertos $\widetilde{V}_i \ni p_i$ para cada $i$ tales que $f(\widetilde{V}_i)$ es abierto y $f|_{\widetilde{V}_i} : \widetilde{V}_i \to f(\widetilde{V}_i)$ un difeomorfismo. Achicando los abiertos si es necesario (usando que $M$ es Hausdorff y localmente conexa) podemos suponer que son conexos y disjuntos dos a dos.

Notando $\widetilde{W} = \bigcap_{i=1}^nf(\widetilde{V}_i)$, definimos $\widetilde{U}_i := \widetilde{V}_i \cap f^{-1}(\widetilde{W})$. Ahora, como $M$ es compacta y $N$ es Hausdorff (pues es una variedad) sabemos que $f$ es cerrada. Por lo tanto $f\left((\bigsqcup_{i=1}^n\widetilde{U}_i)^c\right)$ es cerrado y podemos definir $W := \widetilde{W} \setminus f\left((\bigsqcup_{i=1}^n\widetilde{U}_i)^c\right)$ y $U_i := \widetilde{U}_i \cap f^{-1}(W)$.

De aquí se ve que $f^{-1}(W) = U_1 \sqcup \cdots \sqcup U_k$ ya que es
\begin{align*}
f^{-1}(W) &= f^{-1}\left(\widetilde{W} \setminus f\left(\left(\bigsqcup_{i=1}^n\widetilde{U}_i\right)^c\right)\right) = \big(f|^{\widetilde{W}}\big)^{-1}\left(f\left(\left(\bigsqcup_{i=1}^n\widetilde{U}_i\right)^c\right)^c\right)\\
&= \big(f|^{\widetilde{W}}\big)^{-1}\left(f|_{\widetilde{W}}\left(\left(\bigsqcup_{i=1}^n\widetilde{U}_i\right)^c\right)\right)^c \subset \left(\bigsqcup_{i=1}^n\widetilde{U}_i\right)^{cc} = \bigsqcup_{i=1}^n\widetilde{U}_i 
\end{align*}
de forma que
\begin{align*}
f^{-1}(W) \subset f^{-1}(W) \cap \bigsqcup_{i=1}^n\widetilde{U}_i  = \bigsqcup_{i=1}^n\widetilde{U}_i \cap f^{-1}(W) = \bigsqcup_{i=1}^nU_i,  
\end{align*}
y la otra contención está dada por la definición de los abiertos $(U_i)_{i \in \nat{n}}$. 

Por último, para ver que que cada (co)restricción $f|_{U_i}^W$ es un difeomorfismo basta ver que $f(U_i) = W$ pues ya sabemos que la (co)restricción es suave e inyectiva al serlo $f|_{\widetilde{V_i}}$. Por como definimos $U_i$, basta ver que $W \subset f(U_i)$. 

En efecto, puesto que es $W \subset \widetilde{W} \subset f(\widetilde{V}_i)$, si $x \in W$ entonces existe $v_i \in \widetilde{V}_i$ con $f(v_i) = x$, y dado que además es $v_i \in f^{-1}(W) \subset f^{-1}(\widetilde{W})$, se obtiene que 
\begin{align*}
v_i \in \widetilde{V}_i \cap f^{-1}(\widetilde{W}) = \widetilde{U}_i.
\end{align*}
En consecuencia debe ser $v_i \in \widetilde{U}_i \cap f^{-1}(W) = U_i$, y así $x \in f(U_i)$.

\end{proof}

\begin{exercise}{2} Sean $M$ y $N$ dos variedades compactas, conexas, orientadas y de la misma dimensión $n$ y sea $f : M \to N$ una función diferenciable.
\begin{itemize}[listparindent = \parindent]
\item[(a)] Muestre que hay un número real $\lambda \in \R$ tal que para toda forma $\omega \in \Omega^n(N)$ se tiene 
\begin{align*}
\int_Mf^*(\omega) = \lambda\int_N \omega.
\end{align*}
Lo llamamos \textit{grado} de $f$ y lo escribimos $\deg(f)$.
\item[(b)] Supongamos que $q \in N$ es valor regular de $f$, de manera que, en particular, el conjunto $f^{-1}(q)$ es finito. Si $p \in f^{-1}(q)$ la diferencial es entonces un isomorfismo de espacios vectoriales y podemos considerar el número
\begin{align*}
sgn_f(p) = \begin{cases}
+1&\text{ si $d_pf$ preserva la orientación}\\
-1&\text{si la invierte}
\end{cases}
\end{align*}
ya que esas son las dos únicas posibilidades.

Muestre que
\begin{align*}
\deg(f) = \sum_{p \in f^{-1}(q)}sgn_f(p).
\end{align*}
\end{itemize}
\end{exercise}
\begin{proof} Observemos en primer lugar una consecuencia del $\tpaing{Teorema $1$}$ que usaremos a continuación: este nos dice que en una variedad orientada conexa y compacta, dos $n$-formas son cohomólogas si y sólo si sus integrales sobre la variedad coinciden.
\begin{itemize}[listparindent = \parindent] 
\item[(a)] En vista del $\tpaing{Teorema $1$}$, sabemos que existe una $n$-forma $\omega$ de $N$ tal que $\int_N \omega = 1$. De existir, el grado debe cumplir que $\int_M f^*(\omega) = \deg(f) \cdot \int_N \omega = \deg(f)$, por lo que definimos
\begin{align*}
\deg(f) := \int_M f^*(\omega).
\end{align*}

Además $\deg(f)$ está bien definido pues no depende de la forma que elegimos: si $\eta$ es otra $n$-forma de $N$ que integra $1$, entonces es $[\eta] = [\omega]$ y por tanto $[f^*(\omega)] = [f^*(\eta)]$, de lo que resulta $\int_M f^*(\omega) = \int_M f^*(\eta)$. 

Veamos ahora que este número satisface la propiedad deseada. Consideremos una forma $n$-forma abritraria $\eta$ de $N$. Una vez más por el $\tpaing{Teorema $1$}$, del isomorfismo $H^n(N) \simeq \R$ sabemos que existe $c \in \R$ de forma que $[\eta] = c[\omega] = [c\omega]$. En consecuencia existe $\kappa \in \Omega^{n-1}(N)$ tal que $\eta - c\omega = d\kappa$ y usando el teorema de Stokes\footnote{También podríamos apelar de vuelta al $\tpaing{Teorema $1$}$. De hecho, una pequeña modificación de la cuenta que sigue justifica que la aplicación del teorema está bien definida.} es
\begin{align*}
\int_N \eta = \int_N c \cdot \omega + d\kappa = c\cdot\int_N\omega + \int_Nd\kappa = c\cdot\int_N\omega + \int_{\partial N}i^*(\kappa) = c
\end{align*}  
ya que $N$ no tiene borde. 

Por otro lado, como
\begin{align*}
f^*(\eta) = f^*(c\cdot\omega + d\kappa) = c\cdot f^*(\omega) + f^*(d\kappa) = c\cdot f^*(\omega) + df^*(\kappa)
\end{align*}
y $M$ tampoco tiene borde, volvemos a apelar al teorema de Stokes para obtener
\begin{align*}
\int_M f^*(\eta) &= c\cdot\int_M f^*(\omega) + \int_M df^*(\kappa) = c \cdot \deg(f) + \int_{\partial M}i^*(f^*(\kappa))\\
&= c \cdot \deg(f) = \deg(f) \cdot \int_N \eta
\end{align*}
como buscábamos.
\item[(b)] Será de utilidad la siguiente observación: para cualquier abierto $U \subset N$, existe una $n$-forma $\omega$ de $N$ tal que $1 = \int_N \omega = \int_U \omega$. 

Para construirla, consideramos primero  $\eta$ una forma de volumen, una carta $(V,\varphi)$ tal que $V \subset U$ y una función \textit{bump} $h \in C^\infty$ no negativa que satisfaga $\sop h \subset V$. De esta forma, en el abierto $V$ la forma $\eta$ tiene una escritura en coordenadas $\eta = g \cdot d\varphi^1 \wedge \cdots \wedge d\varphi^n$ con $g$ nunca nula. 

Definimos ahora $\widetilde{\omega} = h\eta$. Como $\sop(\widetilde{\omega}) \subset \sop h \subset U$, es $\int_N \widetilde{\omega} = \int_U \widetilde{\omega}$. Además, en $V$ la forma $\widetilde{\omega}$ tiene una expresión de la forma $hg \cdot d\varphi^1 \wedge \cdots \wedge d\varphi^n$ así que
\begin{align*}
\int_U \widetilde{\omega} = \int_{V} hg \ d\varphi^i \wedge \cdots \wedge d\varphi^n = \int_{\varphi(V)} (hg) \circ \varphi^{-1} \ dx_1 \dots dx_n = \int_{\sop h} (hg) \circ \varphi^{-1} \ dx_1 \dots dx_n > 0
\end{align*}
pues $(hg) \circ \varphi^{-1}$ es continua, nunca nula en $\sop h$ y no cambia de signo en $V \supset \sop h$. Basta tomar entonces $\omega := (\int_N \widetilde{\omega})^{-1} \cdot \widetilde{\omega}$.

Ahora sí, analicemos primero qué ocurre cuando $q \not \in \im f$. Como $M$ es compacta y $N$ es Hausdorff (pues es una variedad), la función $f$ es cerrada. Por lo tanto la imagen de $f$ es un cerrado, y en consecuencia existe un abierto $U \ni q$ contenido en $f(M)^c$. 

Podemos tomar entonces una $n$-forma $\omega$ tal que $\sop \omega \subset U$ y $\int_N \omega = 1$. Como $\sop f^*(\omega) \subset  f^{-1}(\sop \omega) = \emptyset$, es
\begin{align*}
\deg(f) = \deg(f) \int_N\omega = \int_Mf^*(\omega) = 0
\end{align*}
lo que nos dice que 
\begin{align*}
\deg(f) = 0 = \sum_{p \in f^{-1}(q)}sgn_f(p).
\end{align*}

Ahora sí, supongamos que $f^{-1}(q) = \{p_1, \dots, p_k\} \subset M$ con $n \geq 1$. Usando el $\tpaing{Lema $3$}$, tomemos un abierto $W$ de $N$ y abiertos conexos disjuntos $U_i \ni p_i$ tales que $f^{-1}(W) = \bigsqcup_{i=1}^k U_i$ y cada (co)restricción $f|_{U_i}^W$ es un difeomorfismo. 

En particular, como $sgn_f(p)$ varía suavemente con $p$ y tiene codominio discreto, en cada abierto conexo $U_i$ debe ser constante. Es decir, para cada $i \in \nat{n}$ la función $f$ preserva o invierte la orientación en \textit{todo} el abierto $U_i$.

Una vez más, podemos considerar una $n$-forma $\omega$ tal que $\sop \omega \subset W$ y $1 = \int_W \omega= \int_M \omega$. Por lo tanto, es
\begin{align*}
\deg(f) &= \int_Mf^*(\omega) = \int_{f^{-1}(\sop \omega)}f^*(\omega) = \int_{f^{-1}(W)}f^*(\omega)\\
&= \int_{\bigsqcup_{i=1}^nU_i}f^*(\omega) = \sum_{i=1}^k \int_{U_i}f^*(\omega).
\end{align*}

Usando el $\tpaing{Teorema $2$}$ para cada difeomorfismo $f|_{U_i}^W : U_i \to W$ y recordando que $sgn_f$ es constante en $U_i$, es
\begin{align*}
\int_{U_i}f^*(\omega) = sgn_f(p_i) \cdot \int_W\omega = sgn_f(p_i).
\end{align*}
Se obtiene así
\begin{align*}
\deg(f) = \sum_{i=1}^k \int_{U_i}f^*(\omega) = \sum_{i=1}^k sgn_f(p_i).
\end{align*}
\end{itemize}
\end{proof}

\paintline
\newpage
\begin{lemma}{4} Sea $\eta \in \Ss^1$ una $1$-forma y $n \in \N$. Si definimos $\omega = \pi^\ast_1(\eta) \wedge \cdots \wedge \pi^\ast_n(\eta) \in \Omega^n(\T^n)$ con $\pi_i : \T^n \to \Ss^1$ la proyección a la $i$-ésima coordenada, entonces
\begin{align*}
\int_{\T^n}\omega = \left(\int_{\Ss^1} \eta\right)^n.
\end{align*}
\end{lemma}
\begin{proof} Dadas proyecciones estereográficas $\{\varphi_i : U_i \to \R \}_{i=1}^n$ de $\Ss^1$, tenemos una carta
\begin{align*}
\Psi := \varphi_1 \times \cdots \times \varphi_n : U_1 \times \dots \times U_n \to \R^n
\end{align*}
de $\T^n$ que satisface $\pi_i(\Psi) = \varphi_i$ para cada $i \in \nat{n}$. Más aún\footnote{Esta es una cuenta muy similar al ejercicio $(1)$ de la práctica $1$, que corresponde a la primera entrega.}, para cada $i \in \nat{n}$ es
\begin{align*}
d_p\pi_i\left(\frac{\partial}{\partial \Psi^j}\Big|_p\right) = \delta_{ij} \cdot \frac{\partial}{\partial \varphi_i}\Big|_{p_i}.
\end{align*}

A partir de esto último, afirmamos que $\pi_i^*(d\varphi_i) = d\Psi^i$. Basta probar que en cada punto $p \in U_1 \times \cdots \times U_n$ ambas $1$-formas coinciden en una base de $T_p\T^n$. Tomando los \textit{ganchos} $\{\frac{\partial}{\partial \Psi^i}\big|_p\}$, efectivamente es
\begin{align*}
(\pi_i)_p^*(d\varphi_i)\left(\frac{\partial}{\partial \Psi^j}\Big|_p\right) &= d_{p_i}\varphi_i\left(d_p\pi_i\left(\frac{\partial}{\partial \Psi^j}\Big|_p\right) \right) = d_{p_i}\varphi_i\left(\delta_{ij} \cdot \frac{\partial}{\partial \varphi_i}\Big|_{p_i}\right)
\\&= \delta_{ij}d_{p_i}\varphi_i\left(\frac{\partial}{\partial \varphi_i}\Big|_{p_i}\right) = \delta_{ij} \cdot \delta_{ij}\\
&= \delta_{ij} = d_p\Psi^i\left(\frac{\partial}{\partial \Psi^j}\Big|_p\right).
\end{align*}

Como $\eta$ es una $1$-forma, para cada $i \in \nat{n}$ existe una función suave $g_i \in C^\infty(\Ss^1)$ tal que $\eta = g_i \cdot d\varphi_i$, y se tiene\footnote{Estas propiedades son parte del ejercicio $(8)$ de la práctica $4$, que elegí resolver para la cuarta entrega.} entonces que $\pi_i^\ast(\eta) = g_i \circ \pi_i \cdot \pi_i^*(d\varphi_i) = g_i \circ \pi_i \cdot d\Psi^i$. Reescribiendo, tenemos una expresión para $\omega$ en terminos de $d\Psi^1, \dots, d\Psi^n$,
\begin{align*}
\omega = \pi^\ast_1(\eta) \wedge \cdots \wedge \pi^\ast_1(\eta) = g_1 \circ \pi_1 \cdot d\Psi^1 \wedge \cdots \wedge g_n \circ \pi_n \cdot d\Psi^n = \prod_{i = 1}^{n}g_i \circ \pi_i \cdot d\Psi^1 \wedge \cdots \wedge d\Psi^n.
\end{align*}

Tanto $\Psi$ como cada carta $\psi_i$ tienen dominio denso cuyo complemento es de medida cero, así que en vista de las anteriores caracterizaciones de $\omega$ y $\eta$ podemos calcular sus integrales como
\begin{align*}
\int_{\T^n}\omega = \int_{\R^n}\prod_{i = 1}^{n}g_i \circ \pi_i \cdot dx_1 \cdots dx_n \quad  \text{ y } \quad \int_{\Ss^1} \eta = \int_{\R}g_i(x)dx
\end{align*}
para cada $i \in \nat{n}$.

Finalmente usando el teorema de Fubini, es
\begin{align*}
\int_{\T^n}\omega &= \int_{\R^n}\prod_{i = 1}^{n}g_i \circ \pi_i(x_1, \dots, x_n) \cdot dx_1 \cdots dx_n = \int_{\R^n}\prod_{i = 1}^{n}g_i(x_i) \cdot dx_1 \cdots dx_n\\
&= \left(\int_{\R}g_1(x_1)dx_1\right) \cdots \left(\int_{\R}g_n(x_n)dx_n\right)\\
&= \left(\int_{\Ss^1} \eta\right)^n.
\end{align*}
\end{proof}

\begin{exercise}{3} Muestre que cuando $n \geq 2$ el grado de toda función diferenciable $f : \Ss^n \to \T^n$ de la $n$-esfera al $n$-toro es nulo.
\end{exercise}
\begin{proof} Consideremos $\eta \in \Omega^1(\Ss^1)$ una forma de volumen. Definimos entonces
\begin{align*}
\omega = \pi^\ast_1(\eta) \wedge \cdots \wedge \pi^\ast_n(\eta) \in \Omega^n(\T^n)
\end{align*}
con $\pi_i : \T^n \to \Ss^1$ la proyección a la $i$-ésima coordenada. 

Al ser $n \geq 2$ sabemos que $H^1(\Ss^n) = 0$, y por lo tanto para todo $i \in \nat{n}$ resulta $[f^*\pi_i^*(\eta)] = 0 \in H^1(\Ss^n)$. Como $f^* : H^\bullet(\T^n) \to H^\bullet(\Ss^n)$ es un morfismo de álgebras, esto dice que
\begin{align*}
[f^*(\omega)] &= f^*([\omega]) = f^*([\pi^\ast_1(\eta) \wedge \cdots \wedge \pi^\ast_1(\eta)])\\
&= f^*([\pi^\ast_1(\eta)]) \wedge \cdots \wedge f^*([\pi^\ast_1(\eta)])\\
&= [0] \wedge \cdots \wedge [0] = [0].
\end{align*}

Vemos así que $f^*(\omega)$ es exacta y por lo tanto, existe $\zeta \in \Omega^{n-1}(\T^n)$ tal que $f^*(\omega) = d\zeta$. Por el teorema de Stokes, esto implica que
\begin{align*}
\int_{\Ss^n}f^*(\omega) = \int_{\Ss^n}d\zeta = \int_{\partial \Ss^n}i^*(\zeta) = 0
\end{align*}
ya que $\Ss^n$ no tiene borde.

Del $\tpaing{Lema $4$}$ y la definición de grado, obtenemos
\begin{align*}
0 = \int_{\Ss^n}f^*(\omega) = \deg(f) \cdot \int_{\T^n}\omega = \deg(f) \cdot \left(\int_{\Ss^1} \eta\right)^n.
\end{align*}
Al ser $\eta$ una forma de volumen de $\Ss^1$, su integral es no nula, y consecuentemente es $\deg(f) = 0$. 
\end{proof}

\paintline
\hspace{25pt}
\begin{tcolorbox}
\begin{theorem}{5 (dualidad de Poincaré)} Sea $M$ una variedad conexa y orientable de dimensión $n$. Entonces, para cada $k \in \nat{n}$ se tiene que $H_c^k(M) \simeq H^{n-k}(M)^*$ y el isomorfismo está dado por
\begin{align*}
p : H_c^k(&M) \longrightarrow H^{n-k}(M)^*\\
&[\omega] \mapsto \left([\eta] \mapsto (-1)^{|\omega|}\int_M \omega \wedge \eta\right)
\end{align*}
\end{theorem}
\end{tcolorbox}
\newpage
\begin{exercise}{4} Sea $M$ una variedad compacta, orientable, conexa de dimensión $n$. Sabemos que la cohomología de De Rham de $M$ tiene entonces dimensión total finita, y podemos en consecuencia considerar el entero
\begin{align*}
\chi(M) = \sum_{i=0}^{n}(-1)^i \dim H^i(M)
\end{align*}
al que llamamos la \textit{característica de Euler} de $M$.
\begin{itemize}[listparindent = \parindent]
\item[(a)] Si la dimensión $n$ de $M$ es impar, entonces $\chi(M) = 0$.
\item[(b)] Si la dimensión $n$ de $M$ es par y la de $H^{n/2}(M)$ es par, entonces $\chi(M)$ es un entero par.
\end{itemize}
\end{exercise}
\begin{proof} Notemos que como la variedad $M$ es compacta, su cohomología a soporte compacto coincide con la cohomología de de Rham «a secas». Además, dado que la cohomología de una variedad compacta es finitamente generada, por el $\tpaing{Teorema $5$}$ se tiene que
\begin{align*}
H^i(M) = H^i_c(M) \simeq H^{n-i}(M)^* \simeq H^{n-i}(M) 
\end{align*}
para cada $i \in \natzero{n}$. En particular, si notamos $\beta_i := \dim H^i(M)$ a la dimensión del $i$-ésimo grupo de cohomología, debe ser $\beta_i = \beta_{n-i}$.
Ahora,
\begin{itemize}[listparindent = \parindent]
\item[(a)] Si $n$ es impar, entonces
\begin{align*}
2\chi(M) &= \sum_{i=0}^{n}(-1)^i\beta_i + \sum_{i=0}^{n}(-1)^i\beta_i =
\sum_{i=0}^{n}(-1)^i\beta_i + \sum_{i=0}^{n}(-1)^{n-i}\beta_{n-i}\\
&= \sum_{i=0}^{n}(-1)^i\beta_i + (-1)^n\sum_{i=0}^{n}(-1)^{-i}\beta_{n-i}\\
&= \sum_{i=0}^{n}(-1)^i\beta_i + (-1)^n\sum_{i=0}^{n}(-1)^{i}\beta_{i}\\
&= \chi(M)(1 + (-1)^n) = \chi(M) \cdot  0 = 0.
\end{align*}
Por lo tanto, es $\chi(M) = 0$.
\item[(b)] De una forma similar, si $n$ y es par y $\beta_{n/2}$ también, entonces
\begin{align*}
\chi(M) &= \sum_{i=0}^{n}(-1)^i\beta_i = \sum_{i=0}^{n/2-1}(-1)^i\beta_i + \beta_{n/2} + \sum_{i=n/2+1}^{n}(-1)^i\beta_i\\
&= \sum_{i=0}^{n/2-1}(-1)^i\beta_i + \beta_{n/2} + \sum_{i=0}^{n/2-1}(-1)^{n-i}\beta_{n-i}\\
&= 2\sum_{i=0}^{n/2-1}(-1)^i\beta_i + \beta_{n/2} \equiv \beta_{n/2} \equiv 0 \pmod{2}.
\end{align*}
En consecuencia, la característica de Euler de $M$ es par.
\end{itemize}
\end{proof}

\begin{tcolorbox}
\begin{theorem}{6 (fórmula de Koszul)} Si $(M,g)$ es una variedad riemanniana y $\nabla$ su conexión de Levi-Civita, entonces para todo $X,Y,Z \in \X(M)$ es
\begin{align*}
2\ip{\nabla_XY,Z} = X\ip{Y,Z} + Y\ip{Z,X} - Z\ip{X,Y} - \ip{Y,[X,Z]} - \ip{X,[Y,Z]} + \ip{[X,Y],Z}.
\end{align*}
\end{theorem}
\ \\
\end{tcolorbox}

\begin{exercise}{5} Sea $G$ un grupo de Lie de dimensión $n$, sea $\mathfrak{g} = T_eG$ su álgebra de Lie y fijemos un producto interno $g_e : \mathfrak{g} \times \mathfrak{g} \to \R$ sobre $\mathfrak{g}$.
\begin{itemize}[listparindent = \parindent]
\item[(a)] Hay una única métrica riemanniana $g$ sobre $G$ que es invariante a izquierda y cuyo valor en $e \in G$ es $g_e$.
\item[(b)] Sea $\mathscr{B} = \{v_1, \dots, v_n\}$ una base de $\mathfrak{g}$ y sean $X_1, \dots, X_n$ los campos tangentes a $G$ invariantes a izquierda que extienden a los elementos de $\mathscr{B}$. Muestre que para cada $i,j$, es \textit{constante} la función $g_{i,j} = g(X_i,X_j)$.

Sabemos (porque el corchete de Lie de campos invariantes a izquierda es él mismo invariante a izquierda) que existen constantes $c_{i,j}^k$ tales que
\begin{align*}
[X_i,X_j] = \sum_{k}c_{i,j}^kX_k.
\end{align*}
Calcule en términos de los escalares $c_{i,j}^k$ y $g_{i,j}$ los símbolos de Christoffel $\Gamma_{ij}^k$ de la conexión de Levi-Civita de G con respecto a los campos $X_1, \dots, X_n$, de manera que se tenga
\begin{align*}
\nabla_{X_i}X_j = \sum_k\Gamma_{ij}^kX_k.
\end{align*}
\item[(c)] Sea $G = \R_{>0 } \times \R$ el grupo de Lie con producto dado por
\begin{align*}
(a,b) \cdot (c,d) = (ac,ad+b)
\end{align*}
para cada $(a,b),(c,d) \in G$, de manera que $G$ es isomorfo de la forma evidente al grupo de matrices 
\begin{align*}
\left\{\begin{pmatrix}
a & b \\
0 & 1
\end{pmatrix} : a >0, b \in \R\right\}.
\end{align*}
El elemento neutro $G$ es $e = (1,0)$ y su álgebra de Lie $\mathfrak{g} = T_eG$ se identifica de manera natural (porque $G$ es un abierto de $\R^2$) con $\R^2$. Dotemos a $G$ de su única métrica invariante a izquierda que en $T_eG$ restringe al producto interno usual de $\R^2$. Encuentre todas las geodésicas que pasan por $e$ que pueda.

Calcule (las componentes en una carta del) tensor de curvatura $R(X,Y)Z$ sobre $G$ y la \textit{curvatura escalar}
\begin{align*}
K(p) = \frac{1}{n(n-1)} \sum_{1 \leq i,j \leq n}g(R(z_i,z_j)z_i,z_j)
\end{align*}
para cada $p \in G$, con $\{z_1, \dots, z_n\}$ una base ortonormal de $T_pG$.
\end{itemize}
\end{exercise}
\begin{proof} Hago cada inciso por separado. 
\begin{itemize}[listparindent = \parindent]
\item[(a)] Definimos
\begin{align*}
g_h(v,w) := g_e(d_hL_{h^{-1}}(v),d_hL_{h^{-1}}(w)) \in \R
\end{align*}
para cada $h \in G$ y $v,w \in T_hG$, que es suave pues es composición de funciones suaves.  

Probamos primero que $g$ es una métrica, es decir, que $g_h$ es un producto interno en $T_hG$ para cada punto $h$ de la variedad. Fijamos entonces $h \in G$. Como el diferencial $d_hL_{h^{-1}}$ es una función lineal, así lo es $d_hL_{h^{-1}} \times d_hL_{h^{-1}} : T_hG \times T_hG \to \mathfrak{g} \times \mathfrak{g}$. Al ser $g_e$ un producto interno, se sigue que $g_h = g_e \circ (d_hL_{h^{-1}} \times d_hL_{h^{-1}})$ es bilineal, y más aún es simétrica pues 
\begin{align*}
g_h(v,w) = g_e(d_hL_{h^{-1}}(v),d_hL_{h^{-1}}(w)) = g_e(d_hL_{h^{-1}}(w),d_hL_{h^{-1}}(v)) = g_h(w,v).
\end{align*}
Por último, $g_h$ es definida positiva: si $v \in T_hG$, entonces
\begin{align*}
g_h(v,v) = g_e(d_hL_{h^{-1}}(v),d_hL_{h^{-1}}(v)) > 0
\end{align*}
y como $d_hL_{h^{-1}}$ es un isomorfismo lineal,
\begin{align*}
g_h(v,v) = 0\iff g_e(d_hL_{h^{-1}}(v),d_hL_{h^{-1}}(v)) = 0 \iff d_hL_{h^{-1}}(v) = 0\iff v = 0.
\end{align*}

Esto prueba la existencia de una tal métrica. Si $\widetilde{g}$ es otra métrica invariante a izquierda que vale $g_e$ en la identidad, entonces como
\begin{align*}
d_eL_h \circ d_hL_{h^{-1}} = d_h(L_h \circ L_{h^{-1}}) = d_h(id_G) = id_{T_hG} 
\end{align*}
por invariancia es
\begin{align*}
\widetilde{g}_h(v,w) &= \widetilde{g}_{he}(d_eL_h(d_hL_{h^{-1}}(v)),d_eL_h(d_hL_{h^{-1}}(w)))= \widetilde{g}_e(d_hL_{h^{-1}}(v),d_hL_{h^{-1}}(w))\\&= g_e(d_hL_{h^{-1}}(v),d_hL_{h^{-1}}(w)) = g_h(v,w),
\end{align*}
lo que prueba la unicidad.
\item[(b)] Recordemos que por definición es $(X_i)_h = d_eL_h(v_i)$ para cada $i \in \nat{n}$ y $h \in G$. Por lo tanto, usando la invariancia a izquierda de la métrica es
\begin{align*}
(g(X_i,X_j))_h = g_h((X_i)_h,(X_j)_h) = g_h(d_eL_h(v_i)),d_eL_h(v_j)) = g_e(v_i,v_j)
\end{align*}
para todo $h \in G$. Esto muestra que $\ip{X_i,X_j} := g(X_i,X_j)$ vale constantemente $g_{i,j} := g_e(v_i,v_j)$. 

En particular, sabemos que para todo campo $Z \in \X(M)$ es $Z \ip{X_i,X_j} \equiv 0$. Usando esto y la fórmula de Koszul, se obtiene que
\begin{align*}
2\ip{\nabla_{X_i}X_j,X_s} &= \ip{X_s,[X_i,X_j]} - \ip{X_j,[X_i,X_s]} - \ip{X_i,[X_j,X_s]}\\
&= \sum_l\ip{X_s,c^l_{i,j}X_l} - \sum_l\ip{X_j,c^l_{i,s}X_l} - \sum_l\ip{X_i,c^l_{j,s}X_l}\\
&= \sum_l c^l_{i,j}g_{s,l} - \sum_lc^l_{i,s}g_{j,l} - \sum_lc^l_{j,s}g_{i,l}.
\end{align*} 
\newpage
Más compactamente, notando $(\nu_{i,j})_s := \ip{\nabla_{X_i}X_j,X_s}$ y contrayendo índices es

\begin{align*}
(\nu_{i,j})_s = \frac{1}{2}\left(c^l_{i,j}g_{s,l} - c^l_{i,s}g_{j,l} - c^l_{j,s}g_{i,l}\right).
\end{align*}

Por otro lado si $\{\Gamma_{ij}^k\}_{i,j,k}$ son las funciones\footnote{En principio, estas funciones existen por el solo hecho de que los campos $X_1, \dots, X_n$ en cada punto $h \in G$ dan una base de $T_hG$. Sin embargo, veremos \textit{a posteriori} que son suaves.} que satisfacen $\nabla_{X_i}X_j = \sum_k \Gamma_{ij}^k X_k$ en todo punto, haciendo el producto interno contra $X_s$ en ambos lados debe ser
\begin{align*}
(\nu_{ij})_s = \sum_k\Gamma_{ij}^k g_{k,s}
\end{align*}
o dicho de otra forma,
\begin{align*}
\frac{1}{2}\left(c^l_{i,j}g_{s,l} - c^l_{i,s}g_{j,l} - c^l_{j,s}g_{i,l}\right) = \Gamma_{i,j}^kg_{k,s}.
\end{align*}

En conclusión, los símbolos de Christoffel se describen en términos las coordenadas de los productos internos y corchetes de Lie de los campos como

\begin{align*}
\Gamma_{ij}^k = \frac{1}{2}g^{k,s}\left(c^l_{i,j}g_{s,l} - c^l_{i,s}g_{j,l} - c^l_{j,s}g_{i,l}\right).
\end{align*}

\item[(c)] Como la métrica en $g$ extiende el producto interno en la identidad dado por el usual de $\R^2$, es $g_{i,j} = \delta_{ij}$. Por otro lado, si fijamos la base ortonormal de $T_eG$ que corresponde a los \textit{ganchos} $\mathscr{B} = \{\partial_x|_e,\partial_y|_e\}$, veamos como quedan los campos (que notamos $X$ e $Y$ respectivamente) que extienden a estos vectores de forma $G$-invariante. 

Fijado un punto $(x,y)$, la multiplicación por esta a izquierda es
\begin{align*}
L_{(x,y)}(z,w) = \begin{pmatrix}
x & 0\\
0 & x
\end{pmatrix}\cdot \begin{pmatrix}
z\\
w
\end{pmatrix}  + \begin{pmatrix}
0\\y
\end{pmatrix}
\end{align*}
y por lo tanto su diferencial se identifica con la transformación lineal de $\R^2$ que se corresponde con multiplicación por $x \cdot I_2 \in \M{2}{\R}$. Esto nos dice que $X \equiv x \partial_x$ e $Y \equiv x\partial_y$. 

Para conocer los coeficientes $c_{ij}^k$, calculamos los corchetes de Lie de $X$ e $Y$. Ya sabemos que $[X,X] = [Y,Y] = 0$ y por antisimetría es $-[Y,X] = [X,Y]$, así que calculamos este último,
\begin{align*}
[x\partial_x,x\partial_y] &= \partial_x(x)\partial_y + x[\partial_x,x\partial_y]\\
&= \partial_x(x)\partial_y - x[x\partial_y,\partial_x] = \partial_x(x)\partial_y -x\left(\partial_y(x)\partial_x + x[\partial_y,\partial_x]\right)\\
&= \partial_x(x)\partial_y -x\partial_y(x)\partial_x = \partial_y.
\end{align*}
\newpage
Por lo tanto se tiene\footnote{Notar que $(x,y) \mapsto 1/x$ es suave pues el abierto $G \subset \R^2$ no contiene ningún punto de primera coordenada nula.} $c_{11}^k = c_{22}^k = 0$ para todo $k$ y
\begin{align*}
c_{12} = (0,1/x), \quad c_{21} = (0,-1/x).
\end{align*} 
Volviendo a la expresión de los simbolos de Christoffel, como $g_{i,j} = \delta_{i,j}$ queda
\begin{align*}
\Gamma_{ij}^k = \frac{1}{2}(c^k_{ij}-c^j_{ik}-c^i_{jk}).
\end{align*}

Concretamente,
\begin{align*}
\Gamma_{11} &= (0,0), \quad
\Gamma_{22} = (1/x,0), \text{ y}\\
\Gamma_{12} &= (0,0), \quad
\Gamma_{21} = (0,-1/x).
\end{align*}

Ahora sí, calculo todas las geodésicas posibles alrededor de $e \in G$. Como en este caso la carta global es la identidad, aplicando la ecuación de las geodésicas 
\begin{align*}
(x^k)'' + \frac{1}{2}(\Gamma_{ij}^k + \Gamma_{ji}^k)(x^i)'(x^j)' = 0
\end{align*}
a una curva $\gamma(t) = (x(t),y(t))$ resulta

\end{itemize}
\begin{align*}
\begin{cases}
x \cdot x'' = -(y')^2\\
x\cdot y'' = x'\cdot y'
\end{cases}
\end{align*}

En particual, las curvas que cumplen $x'' = -a^2x$ e $y' = ax$ satisfacen la ecuación, por lo que las curvas
\begin{align*}
\gamma(t) = (c\sin(at)+d\cos(at),d\sin(at)-c\cos(at) + b)
\end{align*}
son soluciones. Además pedimos $(d,b-c) = \gamma(0) = e = (1,0)$ por lo que debe ser $d = 1, c = b$. Por otro lado, es $\gamma'(0) = (ac,a)$. Por unicidad local, tenemos finalmente que la geodésica que pasa por $e$ y tiene velocidad $\lambda_1 \partial_x + \lambda_2 \partial_y$ es
\begin{align*}
\gamma_{\lambda_1,\lambda_2}(t) = (\cos(\lambda_2t),\sin(\lambda_2t))+
\frac{\lambda_1}{\lambda_2}(\sin(\lambda_2t),-\cos(\lambda_2t) + 1)
\end{align*}
si $\lambda_2 \neq 0$ y $\gamma_{\lambda_1}(t) = (\lambda_1t+1,0)$ en caso contrario.

\begin{center}
\tpaing{Me faltó calcular las coordenadas del tensor de curvatura y la curvatura escalar.}
\end{center}
\end{proof}

\paintline
\newpage
\begin{lemma}{7} Sea $M$ una variedad de dimensión $n$ y $\omega, \eta \in \Omega^\bullet(M)$. Si $\omega$ y $\eta$ son cerradas y $\omega$ es exacta, entonces $\omega \wedge \eta$ es exacta. Más aún, si $\omega = d\kappa$ entonces es $\omega \wedge \eta = d((-1)^{|\omega|}\omega \wedge \kappa)$.
\end{lemma}
\begin{proof} Por un cálculo directo, es
\begin{align*}
d((-1)^{|\omega|}\omega \wedge \kappa) &= (-1)^{|\omega|}d\omega \wedge \kappa + (-1)^{|\omega|}(-1)^{|\omega|}\omega \wedge d\kappa\\
&= 0 \wedge \kappa + (-1)^{2|\omega|}\omega \wedge \eta = \omega \wedge \eta.
\end{align*}
\end{proof}

\begin{tcolorbox}
\begin{theorem}{8 (fórmula de Künneth)} Sean $N$ y $M$ dos variedades compactas, conexas y orientables. Entonces, para cada $n \in N$ tenemos un isomorfismo
\begin{align*}
H^n(M \times N) \simeq \bigoplus_{r+s = n}H^r(M) \otimes H^s(N)
\end{align*}
inducido por las funciones bilineales
\begin{align*}
([\omega],[\eta]) \in H^r(M) \times H^s(N) \mapsto [\pi_1^*(\omega) \wedge \pi_2^*(\eta)] \in H^n(M \times N)
\end{align*}
para cada $r,s \in \N_0$ tales que $r+s = n$.
\end{theorem}
\end{tcolorbox}

\begin{exercise}{6} Sea $M$ una variedad compacta y orientable de dimensión $4k$.
\begin{itemize}[listparindent = \parindent]
\item[(a)] Muestre que hay una función bilineal no degenerada y simétrica
\begin{align*}
\beta : H^{2k}(M) \times H^{2k}(M) \to \R
\end{align*}
tal que si $\omega$ y $\eta$ son elementos cerrados de $\Omega^{2k}(M)$ entonces 
\begin{align*}
\beta([\omega],[\eta]) = \int_M \omega \wedge \eta.
\end{align*}
Llamamos la signatura de la forma bilineal $\beta$ la signatura de $M$.
\item[(b)] Determine la signatura de $\Ss^4$, de $\Ss^2 \times \Ss^2$, del toro $T^4$, del espacio proyectivo $P_{\C}^2$ y el producto $P_{\C}^2 \times P_{\C}^2$.
\end{itemize}
\end{exercise}
\begin{proof} Resuelvo cada inciso por separado.
\begin{itemize}[listparindent = \parindent]
\item[(a)] Como la aplicación $(\omega,\eta) \in \Omega(M)^{2k} \times \Omega(M)^{2k} \mapsto \omega \wedge \eta \in \Omega(M)^{4k}$ es bilineal, y la integral (que es existe y es finita para toda forma pues $M$ es compacta y orientable) es lineal, tenemos definida una aplicación
\begin{align*}
b : (\omega,\eta) \in \Omega^{2k}(M) \times \Omega^{2k}(M) \mapsto \int_M \omega \wedge \eta.
\end{align*}
Si $\omega$ y $\eta$ son dos $2k$-formas cerradas y $\alpha, \beta$ son dos $(2k-1)$-formas cualesquiera, se tiene que
\begin{align*}
(\omega+d\alpha) \wedge (\eta + d\beta) = \omega \wedge \eta + \omega \wedge d\beta + d\alpha \wedge \eta + d\alpha \wedge d\beta.
\end{align*}
Como todos los términos del lado derecho excepto el primero son el wedge de dos formas cerradas con una de ellas exacta, por el $\tpaing{Lema $7$}$ en definitiva existe $\kappa \in \Omega^{4k-1}(M)$ tal que 
\begin{align*}
(\omega+d\alpha) \wedge (\eta + d\beta) = \omega \wedge \eta + d\kappa.
\end{align*}
Ahora, aplicando $b$ y usando el teorema de Stokes es
\begin{align*}
b(\omega+d\alpha,\eta+d\beta) = \int_M \omega \wedge \eta + \int_M d\kappa = \int_M \omega \wedge \eta + \overbrace{\int_{\partial M}i^*(\kappa)}^{= \ 0} = b(\omega,\eta)
\end{align*}
ya que $M$ no tiene borde.

Esto termina de mostrar que la restricción de $b$ a las formas cerradas en ambas coordenadas pasa al cociente por la identificacion de formas cohomólogas, induciendo así una aplicación $\R$-bilineal
\begin{align*}
\beta : ([\omega],[\eta]) \in H^{2k}(M) \times H^{2k}(M) \mapsto \int_M \omega \wedge \eta \in \R.
\end{align*}

Veamos ahora que tanto $b$ como $\beta$ son simétricas pues la operación wedge lo es en $\Omega^{2k}(M) \times \Omega^{2k}(M)$. Si $\sigma \in \Ss_{4k}$, entonces $\widetilde{\sigma} = \sigma (1 \ 4k)(2 \ (4k-1)) \cdots (2k \ (2k+1))$ tiene el mismo signo que $\sigma$ ya que ambas permutaciones difieren en $2k$ transposiciones. Además la aplicación $\sigma \mapsto \widetilde{\sigma}$ es inversible pues corresponde a multiplicar a derecha por un elemento de $\Ss_{4k}$.

Por el mismo motivo que antes, de la multilinealidad de las formas es $\sigma \cdot \eta \otimes \omega = (-1)^{2k} \cdot \widetilde{\sigma} \cdot \omega \otimes \eta = \widetilde{\sigma} \cdot \omega \otimes \eta$ para toda $\omega, \eta \in \Omega^{2k}(M)$ y $\sigma \in \Ss_{4k}$. En consecuencia resulta
\begin{align*}
\eta \wedge \omega &= \frac{1}{2k!2k!}\sum_{\sigma \in \Ss_{4k}}(-1)^\sigma \sigma \cdot \eta \otimes \omega = \frac{1}{2k!2k!}\sum_{\sigma \in \Ss_{4k}}(-1)^{\widetilde{\sigma}} \widetilde{\sigma}\cdot \omega \otimes \eta\\
&=  \frac{1}{2k!2k!}\sum_{\sigma \in \Ss_{4k}}(-1)^{\sigma} \sigma\cdot \omega \otimes \eta = \omega \wedge \eta.
\end{align*}

Para terminar, fijemos $[\omega] \in H^{2k}(M)$. Como $(-1)^{|\omega|} = (-1)^{2k} = 1$, el $\tpaing{Teorema $5$}$ nos dice en particular que si $\beta([\omega],-)$ es la función nula, entonces $[\omega] = 0$. En otras palabras, la función bilineal $\beta$ es no degenerada. 

\item[(b)] Calculo cada caso por separado,
\begin{itemize}[listparindent = \parindent]
\item[$\bullet$] $\underline{\Ss^4}$: en este caso sabemos que $H^2(\Ss^4) = 0$, y por lo tanto $\beta$ es nula. En consecuencia, la signatura de $\Ss^4$ es $0$.
\item[$\bullet$] $\underline{\Ss^2 \times \Ss^2}$: en vista de la fórmula de Künneth sabemos que
\begin{align*}
H^2(\Ss^2 \times \Ss^2) \simeq H^2(\Ss^2) \otimes H^0(\Ss^2) \oplus H^0(\Ss^2) \otimes H^2(\Ss^2) \simeq H^2(\Ss^2) \oplus H^2(\Ss^2),
\end{align*}
y más aún una base de $H^2(\Ss^2 \times \Ss^2)$ es $\mathscr{B} = \{[\pi_1^*(\omega)],[\pi_2^*(\omega)]\}$ con $\omega \in H^2(\Ss^2)$ una forma de volumen. Más aún, el mapa de la fórmula de Künneth para grado $2k$ nos dice que $[\pi_1^*(\omega) \wedge \pi_2^*(\omega)]$ es no nula (y usando el $\tpaing{Teorema $1$}$, su integral es por tanto no nula). 

Ahora como $H^4(\Ss^2) = 0$ es
\begin{align*}
[\pi_1^*(\omega)] \wedge [\pi_1^*(\omega)] &= [\pi_1^*(\omega \wedge \omega)] = 0,\\
[\pi_1^*(\omega)] \wedge [\pi_2^*(\omega)] &= [\pi_1^*(\omega) \wedge \pi_2^*(\omega)],\\
[\pi_2^*(\omega)] \wedge [\pi_1^*(\omega)] &= [\pi_2^*(\omega) \wedge \pi_1^*(\omega)] = [\pi_1^*(\omega) \wedge \pi_2^*(\omega)], \text{ y}\\
[\pi_2^*(\omega)] \wedge [\pi_2^*(\omega)] &= [\pi_2^*(\omega \wedge \omega)] = 0,
\end{align*}
notando $a := \int_{\Ss^2 \times \Ss^2}\pi_1^*(\omega) \wedge \pi_2^*(\omega)$ la matriz de $\beta$ en la base $\mathscr{B}$ es
\begin{align*}
[\beta]_{\mathscr{B}} = \begin{pmatrix}
0 & a\\
a & 0\\
\end{pmatrix} = a \cdot \begin{pmatrix}
0 & 1\\
1 & 0\\
\end{pmatrix}.
\end{align*}

Por un cálculo directo, sabemos que $[\pi_1^*(\omega)]+[\pi_2^*(\omega)]$ y $[\pi_1^*(\omega)]-[\pi_2^*(\omega)]$ son autovectores de $[\beta]_{\mathscr{B}}$ de autovalores $a$ y $-a$ respectivamente, por lo que la signatura de $\Ss^2 \times \Ss^2$ es $0$.
\item[$\bullet$] $\underline{\T^4}$: una vez más, usamos la fórmula de Künneth. Vemos de esta forma que 
\begin{align*}
H^2(\T^4) &\simeq H^0(\T^2) \otimes H^2(\T^2) \oplus H^2(\T^2) \otimes H^0(\T^2) \otimes H^1(\T^2) \otimes H^1(\T^2)\\
&\simeq  H^2(\T^2) \oplus H^2(\T^2) \oplus (H^1(\T^2) \otimes H^1(\T^2)).
\end{align*}

Con un argumento similar se tiene que
\begin{align*}
H^2(\T^2) \simeq H^1(\Ss^1) \otimes H^1(\Ss^1) \text{ y }
H^1(\T^2) \simeq H^1(\Ss^1) \oplus H^1(\Ss^1),
\end{align*}
por lo que en definitiva es
\begin{align*}
H^2(\T^4) &\simeq  H^2(\T^2) \oplus H^2(\T^2) \oplus (H^1(\T^2) \otimes H^1(\T^2))\\
&\simeq (H^1(\Ss^1) \otimes H^1(\Ss^1)) \oplus (H^1(\Ss^1) \otimes H^1(\Ss^1)) \oplus [(H^1(\Ss^1) \oplus H^1(\Ss^1)) \otimes (H^1(\Ss^1) \oplus H^1(\Ss^1))].
\end{align*}

Persiguiendo los isomorfismos del $\tpaing{Teorema 7}$ y usando que proyectar de $\T^4$ a una copia de $\T^2$ y luego a la $1$-esfera es como considerar directamente una proyección $\pi_i : \T^4 \to \Ss^1$, obtenemos una base de $H^2(\T^4)$ dada por
\begin{align*}
\mathscr{B} = \{ \ [\pi_i^*(\eta) \wedge \pi_j^*(\eta)] \ \}_{1 \leq i <j \leq 4}
\end{align*}
con $\eta$ una forma de volumen de $\Ss^1$, ordenada según el orden lexicográfico de $ij$.

Como en el caso de $\Ss^2 \times \Ss^2$, sabemos que el wedge de dos elementos que «comparten un índice» de $\mathscr{B}$ es nulo pues $H^2(\Ss^1) = 0$. Para los otros casos, reordenando obtenemos $\pm [\pi_1^*(\eta) \wedge \pi_2^*(\eta) \wedge \pi_3^*(\eta) \wedge \pi_4^*(\eta)]$. Usando la fórmula de Künneth otra vez o apelando al $\tpaing{Lema $4$}$, sabemos que $a := \int_{\T^4} \pi_1^*(\eta) \wedge \pi_2^*(\eta) \wedge \pi_3^*(\eta) \wedge \pi_4^*(\eta) \neq 0$ y entonces por un cálculo directo es
\begin{align*}
[\beta]_{\mathscr{B}} = a \cdot \begin{pmatrix}
0 & 0 & 0 & 0 & 0 & 1\\
0 & 0 & 0 & 0 & -1 & 0\\
0 & 0 & 0 & 1 & 0 & 0\\
0 & 0 & 1 & 0 & 0 & 0\\
0 & -1 & 0 & 0 & 0 & 0\\
1 & 0 & 0 & 0 & 0 & 0
\end{pmatrix}.
\end{align*}

Notando $\omega_{ij} = [\pi_i^*(\eta) \wedge \pi_j^*(\eta)]$ obtenemos que las clases de cohomología $[\omega_{14} + \omega_{23}]$, $[-\omega_{13}+\omega_{24}]$ y $[\omega_{12} + \omega_{34}]$ son autovectores de de $[\beta]_{\mathscr{B}}$ autovalor $a$; y las clases de cohomología $[-\omega_{14} + \omega_{23}]$, $[\omega_{13}+\omega_{24}]$ y $[-\omega_{12} + \omega_{34}]$ son autovectores de $[\beta]_{\mathscr{B}}$ de autovalor $-a$. Consecuentemente, la signatura de $\T^4$ es cero.
\item[$\bullet$] $\underline{P_\C^2}$: voy a usar algunos hechos sobre la chomología del espacio proyectivo complejo. En primer lugar, sabemos que es
\begin{align*}
H^{k}(P_\C^2) = \begin{cases}\R &\text{ si $2 | k$ y $k \leq 4$}\\0 &\text{en caso contrario}\end{cases}
\end{align*}
y más aún, tenemos una $2$-forma $\omega \in H^2(P_\C^2)$ tal que $\omega \wedge \omega$ es una forma de volumen y $a := \int_{P_\C^2}\omega \wedge \omega > 0$. En consecuencia, en la base de $H^2(P_\C^2)$ dada por $\{\omega\}$ la matriz de $\beta$ es simplemente $(a)$. Concluimos entonces que la signatura de $P_\C^2$ es $1$.
\item[$\bullet$] $\underline{P_\C^2 \times P_\C^2}$: en vista de lo anterior y usando que
\begin{align*}
H^4(P_\C^2 \times P_\C^2) \simeq H^4(P_\C^2) \oplus (H^2(P_\C^2) \otimes H^2(P_\C^2)) \oplus H^4(P_\C^2),
\end{align*}
por el mismo argumento que antes vemos que una base de $H^4(P_C^2)$ proviene del pullback $4$-formas y $2$-formas de $P_\C^2$ por cada proyección. Concretamente, si $H^2(P_\C^2) = \ip{[\omega]}$ es la forma del inciso anterior, entonces
\begin{align*}
\mathscr{B}= \{\ [\pi_1^*(\omega \wedge \omega)], [\pi_2^*(\omega \wedge \omega)], [\pi_1^*(\omega) \wedge \pi_2^*(\omega)] \ \}
\end{align*}
es una base de $H^4(P_\C^2 \times P_\C^2)$. Como $[\omega \wedge \omega \wedge \omega] \in H^6(P_\C^2) = 0$, es
\begin{align*}
[\pi_1^*(\omega) \wedge \pi_2^*(\omega)] \wedge [\pi_1^*(\omega \wedge \omega)] = 0 \ \text{ y } \ [\pi_1^*(\omega) \wedge \pi_2^*(\omega)] \wedge [\pi_2^*(\omega \wedge \omega)] = 0.
\end{align*}

Del mismo modo tenemos que 
\begin{align*}
[\pi_1^*(\omega \wedge \omega)] \wedge [\pi_1^*(\omega \wedge \omega)] = [\pi_2^*(\omega \wedge \omega)] \wedge [\pi_2^*(\omega \wedge \omega)] = 0
\end{align*}
pues $H^8(P_\C^2) = 0$. Por último, usando queda
\begin{align*}
 [\pi_1^*(\omega) \wedge \pi_2^*(\omega)] \wedge  [\pi_1^*(\omega) \wedge \pi_2^*(\omega)] &=  [\pi_1^*(\omega) \wedge \pi_1^*(\omega)] \wedge  [\pi_2^*(\omega) \wedge \pi_2^*(\omega)]\\
 &= [\pi_1^*(\omega \wedge \omega)] \wedge  [\pi_2^*(\omega \wedge \omega)]
\end{align*}

Notando $a = \int_{P_\C^2 \times P_\C^2} \pi_1^*(\omega \wedge \omega) \wedge  \pi_2^*(\omega \wedge \omega) > 0$ es
\begin{align*}
[\beta]_{\mathscr{B}} = a \cdot \begin{pmatrix}
0 & 1 & 0\\
1 & 0 & 0\\
0 & 0 & 1
\end{pmatrix},
\end{align*}
que tiene a $ [\pi_1^*(\omega \wedge \omega)] + [\pi_2^*(\omega \wedge \omega)]$ y $ [\pi_1^*(\omega \wedge \omega)] + [\pi_2^*(\omega \wedge \omega)] + [\pi_1^*(\omega) \wedge \pi_2^*(\omega)]$ como autovectores de autovalor $a$ y a $ [\pi_1^*(\omega \wedge \omega)] - [\pi_2^*(\omega \wedge \omega)]$ como autovector de autovalor $-a$. Por lo tanto, la signatura de $P_\C^2 \times P_\C^2$ es $1$.
\end{itemize}
\end{itemize}
\end{proof}

\paintline
\newpage
\begin{exercise}{7} Sea $M \subseteq \R^3$ una superficie orientada y sin borde dotada de su métrica riemanniana inducida por la de $\R^3$ y supongamos que hay un campo vectorial $Z \in \X(M)$ sobre $M$ que no se anula en ningún punto.
\begin{itemize}[listparindent = \parindent]
\item[(a)] Muestre que existe una única forma de elegir campos $X,Y \in \X(M)$ tales que para cada $p \in M$ se tiene que $(X_p,Y_p)$ es una base ortonormal positiva de $T_pM$ y $Z_p = \|Z_p\|X_p$. 
\item[(b)] Hay $1$-formas $\alpha,\beta\in \Omega^1(M)$ tales que $\alpha(X) = \beta(Y) = 1$ y $\alpha(Y) = \beta(X) = 0$. Más aún, existe una forma $\eta \in \Omega^1(M)$ tal que
\begin{align*}
d\alpha = \eta \wedge \beta, \quad d\beta = -\eta \wedge \alpha.
\end{align*}
La forma $\sigma  = \alpha \wedge \beta$ no depende de la elección de $Z$, es una forma de volumen sobre $M$ que determina su orientación y es, de hecho, la forma de volumen riemanniano sobre $M$.
\item[(c)] Existe una función diferenciable $K : M \to \R$ tal que
\begin{align*}
d\eta = -K \cdot \sigma
\end{align*}
y esta función no depende de la elección del campo $Z$.
\item[(d)] Si $M$ es compacta, entonces $\int_M K \cdot \sigma = 0$.
\item[(e)] Usando los resultados anteriores, muestre que no hay sobre $\Ss^2$ un campo vectorial tangente que no se anula en ningún punto.
\end{itemize}
\end{exercise}
\begin{proof} Hago cada inciso por separado.
\begin{itemize}[listparindent = \parindent]
\item[(a)] Antes que nada, recordemos que para cualquier $\R$-espacio vectorial $\mathbb{V}$ de dimensión $2$ con producto interno y vector unitario $v \in \mathbb{V}$, existe un único vector unitario $w \in \mathbb{V}$ tal que $\{v,w\}$ es una base ortonormal orientada positivamente. 

En efecto, al $\ip{v}^\perp$ tener dimensión $1$, está generado por cierto vector $w_0 \in \mathbb{V}$. Luego, de existir $w$ debe ser de la forma $\lambda \cdot w_0$ para cierto $\lambda \in \R$. Como $\{v,w\}$ será orientada positivamente si y sólo si es $1 = \det(v,\lambda w_0)= \lambda \det(v,w_0)$ y $\{v,w_0\}$ es base, podemos tomar $w := \det(v,w_0)^{-1}w_0$. Más aún, el anterior argumento garantiza que esta es la única elección posible.

Volviendo al ejercicio, como el campo $Z$ es nunca nulo, la función $\frac{1}{\|Z\|}$ está bien definida y es suave, y lo mismo ocurre para el campo $X \in \X(M)$ dado por
\begin{align*}
X_p := \frac{Z_p}{\|Z_p\|}
\end{align*}
para cada $p \in M$. Notemos además que este es el único campo posible que satisface $Z \equiv \|Z\|X$. 

Por la observación inicial, $X$ induce entonces un único campo $Y : M \to TM$ tal que $\{X_p,Y_p\}$ es una base ortonormal orientada positivamente de $T_pM$ para cada $p \in M$.
\begin{center}
$\tpaing{Me faltó ver que $Y$ es un campo suave.}$
\end{center}
\item[(b)] Definimos $\alpha := \ip{X,-}$ y $\beta := \ip{Y,-}$. Estas son $1$-formas pues en cada punto $p \in M$ dan una función lineal $v \mapsto \ip{X_p,v}$ ó $v \mapsto \ip{Y_p,v}$ de $T_pM$, y dependen suavemente de los campos y la métrica. Como $\{X,Y\}$ dá una base ortonormal en cada punto, se sigue que $\alpha(X) = \|X\| = 1 = \|Y\| = \beta(Y)$ y $\alpha(Y) = \beta(X) = \ip{X,Y} = 0$. 

Por otro lado, su wedge $\sigma = \alpha \wedge \beta$ satisface
\begin{align*}
\sigma_p(X_p,Y_p) = \alpha(X_p)\beta(Y_p) - \alpha(Y_p)\beta(X_p) = 1
\end{align*}
para cada $p \in M$. Como $\{X_p,Y_p\}$ es una base ortonormal de $T_pM$, la forma $\sigma$ es necesariamente la forma de volumen riemanniano de $M$. En particular, está determinada por la métrica y no depende del campo $Z$.
\begin{center}
$\tpaing{Me faltó ver que $\eta$ existe y no depende de $Z$.}$
\end{center}
\item[(c)] Como para cada punto $p \in M$ las funciones $(d\eta)_p$ y $\sigma_p$ son funciones multilineales alternadas, y $Alt^2(T_pM)$ tiene dimension $1$, sabemos que existe $K_p \in \R$ tal que $-K_p\sigma_p = (d\eta)_p$ para cada $p \in M$. Tenemos bien definida así una función $K : M \to R$. Para terminar, veamos que es suave. 

Fijando un punto $p \in U$ y una carta $(U, \varphi)$ con $U \ni p$, sabemos que en este abierto $\d\eta$ y $\sigma$ se expresan como
\begin{align*}
\sigma = S \cdot \d\varphi_1 \wedge \varphi_2 \ \text{y} \ \eta = E \cdot d\varphi_1 \wedge d\varphi_2
\end{align*}
para ciertas $S,E \in C^\infty(M)$. Más aún, como $\sigma_p \not \equiv 0$ para todo $p \in M$, sabemos que $S$ es nunca nula. Evaluando en cada punto de $U$, por como definimos $K$ debe ser $K = -E/S$, y esta última es una función suave.
\item[(d)] Si $M$ es compacta, la forma $\sigma$ tiene soporte compacto. Por lo tanto, tiene integral finita en $M$ y más aún usando el teorema de Stokes es
\begin{align*}
\int_M K \cdot \sigma = -\int_M (-K \cdot \sigma) = -\int_M d\eta = -\int_{\partial M}i^*(\eta) = 0
\end{align*}
pues $M$ no tiene borde.
\end{itemize}
\end{proof}

\end{document}
