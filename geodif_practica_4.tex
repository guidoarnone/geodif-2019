\documentclass[11pt]{article}
\usepackage[margin=1in]{geometry} 
\usepackage{amsmath,amsthm,amssymb,amsfonts}

\usepackage{mathpazo}
\usepackage{euler}
\usepackage{xcolor}
\usepackage{tikz}
\usepackage{tikz-cd}
\usetikzlibrary{arrows}
\usetikzlibrary{matrix}
\usepackage{fancyhdr}
\pagestyle{fancy}

\newcommand{\N}{\mathbb{N}}
\newcommand{\Z}{\mathbb{Z}}
\newcommand{\Q}{\mathbb{Q}}
\newcommand{\R}{\mathbb{R}}
\newcommand{\C}{\mathbb{C}}
\newcommand{\Ss}{\mathbb{S}}
\newcommand{\M}[2]{\mathsf{M}_{#1}#2}
\newcommand{\im}{\operatorname{im}}
\newcommand{\eps}{\varepsilon}
\newcommand{\dpart}[2]{\frac{\partial#1}{\partial#2}}
\newcommand{\nat}[1]{[\![#1]\!]}
\newcommand{\natzero}[1]{\nat{#1}_0}
\newcommand{\adj}[1]{\operatorname{adj}(#1)}
\newcommand{\ip}[2]{\langle #1 , #2 \rangle}
\newcommand{\ol}{\overline}
\newcommand{\hook}[3]{\frac{\partial}{\partial x_{#1}}\Big\rvert_{#2}^{#3}}
\usepackage{rotating}
\newcommand*{\isoarrow}[1]{\arrow[#1,"\rotatebox{90}{\LARGE{\(\sim\)}}"
]}
\renewcommand{\d}{\operatorname{d}}
\definecolor{color}{RGB}{191, 0, 124}
\newcommand{\paint}[1]{\color{color}{#1}}

\renewcommand*{\proofname}{\paint{Demostraci\'on}}
\newenvironment{theorem}[2][Teorema]{\begin{trivlist}
\item[\hskip \labelsep \paint{{\bfseries #1}}\hskip \labelsep {\bfseries #2.}]}{\end{trivlist}}
\newenvironment{lemma}[2][Lema]{\begin{trivlist}
\item[\hskip \labelsep \paint{{\bfseries #1}}\hskip \labelsep {\bfseries #2.}]}{\end{trivlist}}
\newenvironment{exercise}[2][Ejercicio]{\begin{trivlist}
\item[\hskip \labelsep \paint{{\bfseries #1}}\hskip \labelsep {\bfseries #2.}]}{\end{trivlist}}
\newenvironment{obs}[2][Observaci\'on]{\begin{trivlist}
\item[\hskip \labelsep \paint{{\bfseries #1.}}]}{\end{trivlist}}
\newenvironment{reflection}[2][Resoluci\']{\begin{trivlist}
\item[\hskip \labelsep {\bfseries #1}\hskip \labelsep {\bfseries #2.}]}{\end{trivlist}}
\newenvironment{proposition}[2][Proposici\'on]{\begin{trivlist}
\item[\hskip \labelsep \paint{{\bfseries #1}}\hskip \labelsep {\bfseries #2.}]}{\end{trivlist}}
\newenvironment{corollary}[2][Corolario]{\begin{trivlist}
\item[\hskip \labelsep {\bfseries #1}\hskip \labelsep {\bfseries #2.}]}{\end{trivlist}}

%-----------------------

\title{
\LARGE{\paint{Geometr\'ia Diferencial}}
\\
\vspace{0.5pt}
\small{\paint{Ejercicios para Entregar - Pr\'actica 4}}
}
\author{\paint{Guido Arnone}}
\date{}
\lhead{Guido Arnone}
\rhead{Pr\'actica 4}

\begin{document}

\maketitle

\begin{center}
\paint{\large{Sobre los Ejercicios}}
\end{center}
\begin{center}

\end{center}
\begin{center}
$\paint{
\rule{400pt}{0.5pt}
}$
\vspace{35pt}
\end{center}

\begin{exercise}{6} Si $M$ y $N$ son variedades no vac\'i­as, entonces $M\times N$ es
orientable si y s\'olo si $M$ y $N$ lo son.
\end{exercise}
\begin{proof}
content...
\end{proof}

\begin{exercise}{8} 
\hspace{1pt}
\begin{itemize}
\item[a)] Si $f:M\to N$ es una funci\'on diferenciable entre variedades, probar que los pull-backs $f^*:\Omega^k(N)\to\Omega^k(M)$ son tales que
\begin{itemize}
\item $f^*(\omega_1+\omega_1) = f^*(\omega_1) + f^*(\omega_2)$,
\item $f^*(h\cdot \omega_1) = h\circ f\cdot f^*(\omega_1), \ $ y  
\item $f^*(\omega_1\wedge\omega_2) = f^*(\omega_1)\wedge f^*(\omega_2)$
\end{itemize}
para cada $\omega_1$,~$\omega_2\in\Omega^\bullet(N)$ y $h\in C^\infty(N)$.

\item[b)] Si $U$ y $V$ son abiertos de $\R^n$ y $f:U\to V$ es diferenciable,
entonces
\begin{align*}
  f^*(\d x_i)=\sum_{k=1}^n\frac{\partial f_i}{\partial x_k}\d x_k
\end{align*}
y
\begin{align*}
  f^*(g\cdot\d x_1\wedge\cdots\wedge\d x_n)
        = g\circ f
                \cdot\det\Bigl(\frac{\partial f_i}{\partial x_j}\Bigr)_{i,j}
                \cdot \d x_1\wedge 	\cdots\wedge\d x_n
\end{align*}
para cada $i\in\{1,\dots,n\}$ y cada $g\in C^\infty(V)$.
\end{itemize}
\end{exercise}
\begin{proof}
content...
\end{proof}


\end{document}
