\documentclass[11pt]{article}
\usepackage[margin=1in]{geometry} 
\usepackage{amsmath,amsthm,amssymb,amsfonts}

\usepackage{mathpazo}
\usepackage{euler}
\usepackage{xcolor}
\usepackage{tikz}
\usepackage{tikz-cd}
\usetikzlibrary{arrows}
\usetikzlibrary{matrix}
\usepackage{fancyhdr}
\pagestyle{fancy}

\newcommand{\N}{\mathbb{N}}
\newcommand{\Z}{\mathbb{Z}}
\newcommand{\Q}{\mathbb{Q}}
\newcommand{\R}{\mathbb{R}}
\newcommand{\C}{\mathbb{C}}
\newcommand{\Ss}{\mathbb{S}}
\newcommand{\M}[2]{\mathsf{M}_{#1}#2}
\newcommand{\im}{\operatorname{im}}
\newcommand{\eps}{\varepsilon}
\newcommand{\dpart}[2]{\frac{\partial#1}{\partial#2}}
\newcommand{\nat}[1]{[\![#1]\!]}
\newcommand{\natzero}[1]{\nat{#1}_0}
\newcommand{\adj}[1]{\operatorname{adj}(#1)}
\newcommand{\ip}[2]{\langle #1 , #2 \rangle}
\newcommand{\paint}[2]{\color{#1}{#2}}
\newcommand{\ol}{\overline}
\newcommand{\hook}[3]{\frac{\partial}{\partial x_{#1}}\Big\rvert_{#2}^{#3}}
\usepackage{rotating}
\newcommand*{\isoarrow}[1]{\arrow[#1,"\rotatebox{90}{\LARGE{\(\sim\)}}"
]}
\definecolor{color}{RGB}{155, 132, 0}

\renewcommand*{\proofname}{\paint{color}{Demostraci\'on}}
\newenvironment{theorem}[2][Teorema]{\begin{trivlist}
\item[\hskip \labelsep \paint{color}{{\bfseries #1}}\hskip \labelsep {\bfseries #2.}]}{\end{trivlist}}
\newenvironment{lemma}[2][Lema]{\begin{trivlist}
\item[\hskip \labelsep \paint{color}{{\bfseries #1}}\hskip \labelsep {\bfseries #2.}]}{\end{trivlist}}
\newenvironment{exercise}[2][Ejercicio]{\begin{trivlist}
\item[\hskip \labelsep \paint{color}{{\bfseries #1}}\hskip \labelsep {\bfseries #2.}]}{\end{trivlist}}
\newenvironment{obs}[2][Observaci\'on]{\begin{trivlist}
\item[\hskip \labelsep \paint{color}{{\bfseries #1.}}]}{\end{trivlist}}
\newenvironment{reflection}[2][Resoluci\']{\begin{trivlist}
\item[\hskip \labelsep {\bfseries #1}\hskip \labelsep {\bfseries #2.}]}{\end{trivlist}}
\newenvironment{proposition}[2][Proposici\'on]{\begin{trivlist}
\item[\hskip \labelsep \paint{color}{{\bfseries #1.}}]}{\end{trivlist}}
\newenvironment{corollary}[2][Corolario]{\begin{trivlist}
\item[\hskip \labelsep {\bfseries #1}\hskip \labelsep {\bfseries #2.}]}{\end{trivlist}}

%-----------------------

\title{
\LARGE{\paint{color}{Geometr\'ia Diferencial}}
\\
\vspace{0.5pt}
\small{\paint{color}{Ejercicios para Entregar - Pr\'actica 3}}
}
\author{\paint{color}{Guido Arnone}}
\date{}
\lhead{Guido Arnone}
\rhead{Pr\'actica 3}

\begin{document}

\maketitle

\begin{center}
\paint{color}{\large{Sobre los Ejercicios}}
\end{center}
\begin{center}
Elej\'i resolver los ejercicios $\paint{color}{(2)}$, $\paint{color}{(8)}$ y $\paint{color}{(9)}$.
\end{center}
\begin{center}
$\paint{color}{
\rule{400pt}{0.5pt}
}$
\vspace{35pt}
\end{center}

\begin{exercise}{2} Sean $M$ una variedad y $f\in C^\infty(M)$. Si $f$ tiene un m\'aximo
local en $p\in M$, entonces $f_{*p}=0$.
\end{exercise}
\begin{proof}
content...
\end{proof}

\begin{exercise}{8} Sea $G$ un grupo de Lie, $\mathfrak{g}$ su \'algebra de Lie y $X\in\mathfrak{g}$ un campo vectorial invariante a izquierda. Pruebe que $X$ es \emph{completo} y describa el flujo asociado.
\end{exercise}
\begin{proof}
content...
\end{proof}

\begin{exercise}{9} \item Sea $G$ un grupo de Lie, $e$ su elemento neutro y $\mathfrak{g}$ su \'algebra de Lie. Probar que:
\begin{itemize}
\item Si $v\in T_eG$ es un vector tangente a $G$ en $e$ y $X\in\mathfrak{g}$ es el \'unico campo vectorial invariante a izquierda tal que $X_e=v$, sea $\gamma_v:\R\to G$ la \'unica curva integral de $X$ tal que $\gamma_v(0)=e$. Entonces $\gamma_v$ es un homomorfismo de grupos, esto es,
  \[
  \gamma_v(t+t')=\gamma_v(t)\cdot\gamma_v(t'), \qquad\forall t,t'\in\R.
  \]

\item Definimos una funci\'on $\exp:T_eG\to G$ poniendo, para cada $v\in
T_eG$, 
  \[
  \exp(v)=\gamma_v(1).
  \]
Determine la diferencial $\exp_{*0}:T_eG\to T_eG$ y muestre que $\exp$ es
localmente un difeomorfismo alrededor de~$0$.

\item Muestre que si $v$,~$w\in T_eG$ son tales que $[v,w]=0$, entonces
  \[
  \exp(v+w)=\exp(v)\cdot\exp(w).
  \]
\end{itemize}
\end{exercise}
\begin{proof}
content...
\end{proof}

\end{document}
