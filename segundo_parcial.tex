\documentclass[11pt]{article}
\usepackage[margin=1in]{geometry} 
\usepackage{amsmath,amsthm,amssymb,amsfonts}
\usepackage[utf8]{inputenc}
\usepackage[T1]{fontenc}
\usepackage{microtype}
\usepackage{mathpazo}
\usepackage{euler}
\usepackage{xcolor}
\usepackage{tikz}
\usepackage{tikz-cd}
\usetikzlibrary{arrows}
\usetikzlibrary{matrix}
\usepackage{fancyhdr}
\pagestyle{fancy}
\usepackage{enumitem}

\newcommand{\N}{\mathbb{N}}
\newcommand{\Z}{\mathbb{Z}}
\newcommand{\Q}{\mathbb{Q}}
\newcommand{\R}{\mathbb{R}}
\newcommand{\C}{\mathbb{C}}
\newcommand{\Ss}{\mathbb{S}}
\newcommand{\T}{\mathbb{T}}
\newcommand{\M}[2]{\mathsf{M}_{#1}#2}
\newcommand{\X}{\mathfrak{X}}
\renewcommand{\div}{\operatorname{div}}
\newcommand{\grad}{\operatorname{grad}}
\newcommand{\im}{\operatorname{im}}
\newcommand{\eps}{\varepsilon}
\newcommand{\dpart}[2]{\frac{\partial#1}{\partial#2}}
\newcommand{\nat}[1]{[\![#1]\!]}
\newcommand{\natzero}[1]{\nat{#1}_0}
\newcommand{\adj}[1]{\operatorname{adj}(#1)}
\newcommand{\ip}[1]{\langle #1 \rangle}
\newcommand{\ol}{\overline}
\newcommand{\hook}[3]{\frac{\partial}{\partial x_{#1}}\Big\rvert_{#2}^{#3}}
\usepackage{rotating}
\newcommand*{\isoarrow}[1]{\arrow[#1,"\rotatebox{90}{\LARGE{\(\sim\)}}"
]}
\newcommand{\gancho}[1]{\frac{\partial}{\partial \varphi^{#1}}}
\renewcommand{\d}{\operatorname{d}}
\definecolor{color}{RGB}{0, 17, 102}
\newcommand{\paint}[1]{\color{color}{#1}}
\newcommand{\tpaing}[1]{\paint{\text{#1}}}
\newcommand{\paintline}{\begin{center}
$\paint{
\rule{400pt}{0.5pt}
}$
\vspace{10pt}
\end{center}}

\renewcommand*{\proofname}{\paint{Demostraci\'on}}
\newenvironment{theorem}[2][Teorema]{\begin{trivlist}
\item[\hskip \labelsep \paint{{\bfseries #1}}\hskip \labelsep {\bfseries #2.}]}{\end{trivlist}}
\newenvironment{lemma}[2][Lema]{\begin{trivlist}
\item[\hskip \labelsep \paint{{\bfseries #1}}\hskip \labelsep {\bfseries #2.}]}{\end{trivlist}}
\newenvironment{exercise}[2][Ejercicio]{\begin{trivlist}
\item[\hskip \labelsep \paint{{\bfseries #1}}\hskip \labelsep {\bfseries #2.}]}{\end{trivlist}}
\newenvironment{obs}[2][Observaci\'on]{\begin{trivlist}
\item[\hskip \labelsep \paint{{\bfseries #1.}}]}{\end{trivlist}}
\newenvironment{reflection}[2][Resoluci\']{\begin{trivlist}
\item[\hskip \labelsep {\bfseries #1}\hskip \labelsep {\bfseries #2.}]}{\end{trivlist}}
\newenvironment{proposition}[2][Proposici\'on]{\begin{trivlist}
\item[\hskip \labelsep \paint{{\bfseries #1}}\hskip \labelsep {\bfseries #2.}]}{\end{trivlist}}
\newenvironment{corollary}[2][Corolario]{\begin{trivlist}
\item[\hskip \labelsep {\bfseries #1}\hskip \labelsep {\bfseries #2.}]}{\end{trivlist}}

%-----------------------

\title{
\LARGE{\paint{Geometr\'ia Diferencial}}
\\
\vspace{1pt}
\small\paint{Primer Cuatrimestre -- 2019}
\\
\vspace{0.5pt}
\large{\paint{Segundo Parcial}}
}
\author{\paint{Guido Arnone}}
\date{}
\lhead{Guido Arnone}
\rhead{Segundo Parcial}

\begin{document}

\maketitle

\begin{center}
$\paint{
\rule{400pt}{0.5pt}
}$
\vspace{15pt}
\end{center}

\begin{exercise}{1} Sea $M$ una variedad riemanniana, sea $\nabla$ la conexión de Levi-Civita de $M$ y sea $f : M \to \R$ una función diferenciable.
\begin{itemize}[listparindent = \parindent]
\item[(a)] Muestre que existe un campo vectorial diferenciable $\grad(f) \in \X(M)$ y uno solo con la propiedad de que para cada campo $Y \in \X(M)$ se tiene que
\begin{align*}
\ip{\grad(f), Y} = df(Y).
\end{align*}
Encuentre una expresión en coordenadas para el campo $\grad(f)$.
\item[(b)] La función 
\begin{align*}
X \in \X(M) \mapsto \nabla_X\grad(f) \in \X(M)
\end{align*}
es autoadjunta: cada vez que $X$ e $Y$ son elementos de $\X(M)$ se tiene que
\begin{align*}
\ip{\nabla_X\grad(f),Y} = \ip{X,\nabla_Y\grad(f)}.
\end{align*}
\item[(c)] Muestre que si el campo $\grad(f)$ tiene norma constante, entonces para todo $X \in \X(M)$ se tiene que $\ip{\nabla_{\grad(f)}\grad(f),X} = 0$. Deduzca de esto que bajo esa condición las curvas integrales de $\grad(f)$ son geodésicas.
\end{itemize}
\end{exercise}
\begin{proof} Hago cada inciso por separado.
\begin{itemize}[listparindent = \parindent]
\item[(a)] Notemos en primer lugar que la función $f$ induce una $1$-forma $df$ que en cada punto $p \in M$ vale $d_pf : v \in T_pM \mapsto v(f) \in \R$, el diferencial de $f$ bajo la identificación $T_p\R \simeq \R$. 

Fijemos ahora $p \in M$. Como $d_pf \in (T_pM)^*$ es un elemento del espacio dual de $T_pM$, y este último es un $\R$-espacio vectorial de dimensión finita equipado con un producto interno (inducido por la métrica de $M$), el teorema de representación de Riesz nos asegura que existe un único vector tangente $v_p \in T_pM$ tal que 
\begin{align}
\ip{v_p,w} = d_pf(w) = w(f) \quad (\forall w \in T_pM).
\end{align}
Si definimos $\grad_p(f) := v_p$ para cada $p \in M$, reescribiendo la anterior igualdad es
\begin{align*}
\ip{\grad(f)_p,w} = d_pf(w) \quad (\forall w \in T_pM).
\end{align*}
y éste es el único campo con tal propiedad. En particular, si $Y \in \X(M)$ entonces para cada $p \in M$ es
\begin{align*}
(\ip{\grad(f),Y})_p = \ip{\grad(f)_p,Y_p} = d_pf(Y_p) = (df(Y))_p
\end{align*}
y por lo tanto $\ip{\grad(f),Y} \equiv df(Y)$. 

Para ver la unicidad, recordemos que para cada $p \in M$ y $v \in  T_pM$ existe $Y^v \in \X(M)$ con $Y^v_p = v$. Efectivamente, podemos tomar una carta $(U, \varphi)$ con $U \ni p$ de forma que existan coeficientes $a_1, \dots, a_n$ tales que $v = \sum_{1 \leq i \leq n}a_i \gancho{i}|_p$ y luego tomar el campo
\begin{align*}
Y^v = h \cdot \sum_{i = 1}^na_i \gancho{i}
\end{align*}
con $h \in C^\infty(M)$ una función \textit{bump} (tal que valga $1$ en un entorno abierto $V \subset U$  de $p$ y $0$ en un abierto $W \supset U^c$). 

A partir de esto, podemos concluir que cualquier otro campo $Z \in \X(M)$ que cumpla $\ip{Z,Y} \equiv df(Y)$ para todo $Y \in \X(M)$ deberá satisfacer
\begin{align*}
\ip{Z_p,v} = \ip{Z_p,Y^v_p} = d_pf(Y_p^v) = v(f)
\end{align*}
para todo $p \in M$ y $v \in T_pM$. La unicidad de $\paint{(1)}$ nos dice entonces que $Z_p = v_p = \grad_p(f)$ en todo punto $p \in M$.

Para terminar, veamos una expresión de $\grad_p(f)$ en coordenadas. De aquí se tendrá que el gradiente depende localmente de funciones suaves, y es por lo tanto diferenciable.

Una vez más, fijamos $p \in M$ y consideramos $(\varphi,U)$ una carta de $M$ tal que $U \ni p$. Como los \textit{ganchos} $\{\gancho{1}|_p,\dots,\gancho{n}|_p\}$ son una base de $T_pM$, sabemos que existen únicos coeficientes $c_1, \dots, c_n \in \R$ tales que
\begin{align*}
v_p = \sum_{j = 1}^n c_j \cdot  \gancho{j}\Big|_p.
\end{align*} 
Si ahora tomamos el producto interno de $v_p$ con $\gancho{i}|_p$, es
\begin{align*}
\frac{\partial f}{\partial \varphi^i}\Big|_p = \gancho{i}\Big|_p(f) = \left\ip{v_p,\gancho{i}\Big|_p\right} = \sum_{j = 1}^n c_j \cdot \left\ip{\gancho{j}\Big|_p,\gancho{i}\Big|_p\right} = \sum_{j = 1}^n c_j \cdot (g_p)_{ji}
\end{align*}
para cada $j \in \nat{n}$, y notando $c = (c_1, \dots, c_n)$ esto es equivalente a 
\begin{align*}
c \cdot g_p = \left(\frac{\partial f}{\partial \varphi^1}\Big|_p, \dots, \frac{\partial f}{\partial \varphi^n}\Big|_p\right).
\end{align*}
Por lo tanto debe ser $c = (\frac{\partial f}{\partial \varphi^1}\big|_p, \dots, \frac{\partial f}{\partial \varphi^n}\big|_p)(g_p)^{-1}$ y
\begin{align*}
c_j = \sum_{i=1}^n \frac{\partial f}{\partial \varphi^i}\Big|_p (g_p)^{ij}.
\end{align*}
Volviendo a la expresión original, obtenemos finalmente
\begin{align*}
\grad_p(f) = v_p = \sum_{j = 1}^n \left(\sum_{i=1}^n \frac{\partial f}{\partial \varphi^i}\Big|_p (g_p)^{ij}\right) \cdot \gancho{j}\Big|_p = \sum_{i,j}(g_p)^{ij} \frac{\partial f}{\partial \varphi^i}\Big|_p \cdot \gancho{j}\Big|_p.
\end{align*}

Esto prueba que para todo $p \in U$ se tiene (contrayendo indices) que
\begin{align*}
\grad(f) = g^{ij}\frac{\partial f}{\partial \varphi^i}\gancho{j}.
\end{align*}
Como afirmamos, esto prueba además que $\grad(f)$ es diferenciable, ya que para cada punto tenemos un abierto donde este es un campo suave.
\item[(b)] Como $\nabla$ es la conexión de Levi-Civita, sabemos (por construcción) que esta es compactible con la métrica y libre de torsion. Concretamente, si $X,Y,Z \in \X(M)$ entonces
\begin{align}
X\ip{Y,Z} &= \ip{\nabla_XY,Z} + \ip{Y,\nabla_XZ}
\end{align}
y
\begin{align}
\nabla_XY - \nabla_YX &= [X,Y].
\end{align}
Sean ahora $X,Y \in \X(M)$ dos campos en $M$. Por $\paint{(2)}$ y $\paint{(3)}$ sabemos que
\begin{align*}
\ip{\nabla_X\grad(f),Y} &= X\ip{\grad(f),Y}-\ip{\grad(f),\nabla_XY}\\
&= X\ip{\grad(f),Y} - \ip{\grad(f),[X,Y]+\nabla_YX}\\
&= X\ip{\grad(f),Y} - \ip{\grad(f),\nabla_YX} - \ip{\grad(f),[X,Y]}\\
&= X\ip{\grad(f),Y} - (Y\ip{\grad(f),X}-\ip{\nabla_Y\grad(f),X}) - \ip{\grad(f),[X,Y]}\\
&= \ip{\nabla_Y\grad(f),X} + X\ip{\grad(f),Y} - Y\ip{\grad(f),X} - \ip{\grad(f),[X,Y]}.
\end{align*} 

En consecuencia, se tiene que $\ip{\nabla_X\grad(f),Y} = \ip{X,\nabla_Y\grad(f)}$ si y solo si
\begin{align*}
X\ip{\grad(f),Y} - Y\ip{\grad(f),X} = \ip{\grad(f),[X,Y]}.
\end{align*}
o lo que es lo mismo,
\begin{align*}
Xdf(Y) - Ydf(X) = df([X,Y]).
\end{align*}

Para terminar, observemos que esto ocurre siempre, pues
\begin{align*}
Xdf(Y) - Ydf(X) = XY(f) - YX(f) = (XY-YX)(f) = [X,Y](f) = df([X,Y]).
\end{align*}
\item[(c)] Supogamos que $\grad(f)$ tiene norma constante. Entonces $\|\grad(f)\|^2 = \ip{\grad(f),\grad(f)}$ debe valer constantemente $c$ para cierto $c \in \R$. Si ahora tomamos un campo $X \in \X(M)$, para todo $p \in M$ debe ser
\begin{align*}
(X\|\grad(f)\|)_p = X_p(c) = 0,
\end{align*}
y por lo tanto $X \ip{\grad(f),\grad(f)} \equiv 0$. 

Usando la compatibilidad con la métrica de la conexión de Levi-Civita, de la anterior igualdad se desprende que 
\begin{align*}
0 &= X \ip{\grad(f),\grad(f)} = \ip{\nabla_X\grad(f),\grad(f)} + \ip{\grad(f),\nabla_X\grad(f)}\\
&= 2\ip{\nabla_X\grad(f),\grad(f)}
\end{align*}
y por $\paint{(b)}$ es
\begin{align*}
\ip{\nabla_{\grad(f)}\grad(f),X} = \ip{\nabla_X\grad(f),\grad(f)} = 0.
\end{align*}

Por último, si $\gamma : I \to M$ es una curva integral de $\grad(f)$, entonces es $\nabla_{\dot{\gamma}}\dot{\gamma} \equiv 0$  pues
\begin{align*}
\|\nabla_{\dot{\gamma}(t)}\dot{\gamma}(t)\|^2 &= \ip{\nabla_{\dot{\gamma}(t)}\dot{\gamma}(t),\nabla_{\dot{\gamma}(t)}\dot{\gamma}(t)}\\
& = \ip{\nabla_{\grad(f)_{\gamma(t)}}\grad(f)_{\gamma(t)},\nabla_{\grad(f)_{\gamma(t)}}\grad(f)_{\gamma(t)}}\\
& = (\ip{\nabla_{\grad(f)}\grad(f),\nabla_{\grad(f)}\grad(f)})_{\gamma(t)} = 0
\end{align*}
para todo $t \in I$. Vemos así que las curvas integral de $\grad(f)$ resultan geodésicas.
\end{itemize}
\end{proof}

\paintline

\begin{exercise}{2} Sean $M$ y $N$ dos variedades compactas, conexas, orientadas y de la misma dimensión $n$ y sea $f : M \to N$ una función diferenciable.
\begin{itemize}[listparindent = \parindent]
\item[(a)] Muestre que hay un número real $\lambda \in \R$ tal que para toda forma $\omega \in \Omega^n(N)$ se tiene 
\begin{align*}
\int_Mf^*(\omega) = \lambda\int_N \omega.
\end{align*}
Lo llamamos \textit{grado} de $f$ y lo escribimos $\deg(f)$.
\item[(b)] Supongamos que $q \in N$ es valor regular de $f$, de manera que, en particular, el conjunto $f^{-1}(q)$ es finito. Si $p \in f^{-1}(q)$ la diferencial es entonces un isomorfismo de espacios vectoriales y podemos considerar el número
\begin{align*}
sgn_f(p) = \begin{cases}
+1&\text{ si $d_pf$ preserva la orientación}\\
-1&\text{si la invierte}
\end{cases}
\end{align*}
ya que esas son las dos únicas posibilidades.

Muestre que
\begin{align*}
\deg(f) = \sum_{p \in f^{-1}(q)}sgn_f(p).
\end{align*}
\end{itemize}
\end{exercise}
\begin{proof}
content...
\end{proof}

\paintline

\begin{lemma}{1} Sea $\eta \in \Ss^1$ una $1$-forma y $n \in \N$. Si definimos $\omega = \pi^\ast_1(\eta) \wedge \cdots \wedge \pi^\ast_n(\eta) \in \Omega^n(\T^n)$ con $\pi_i : \T^n \to \Ss^1$ la proyección a la $i$-ésima coordenada, entonces
\begin{align}
\int_{\T^n}\omega = \int_{\Ss^1 \times \cdots \times \Ss^1} \pi^\ast_1(\eta) \wedge \cdots \wedge \pi^\ast_1(\eta) = \left(\int_{\Ss^1} \eta\right)^n.
\end{align}
\end{lemma}
\begin{proof} Dadas proyecciones estereográficas $\{\varphi_i : U_i \to \R \}_{i=1}^n$ de $\Ss^1$, tenemos una carta
\begin{align*}
\Psi := \varphi_1 \times \cdots \times \varphi_n : U_1 \times \dots \times U_n \to \R^n
\end{align*}
de $\T^n$ que satisface $\pi_i(\Psi) = \varphi_i$ para cada $i \in \nat{n}$. Más aún\footnote{Esta es una cuenta muy similar al ejercicio $(1)$ de la práctica $1$, que corresponde a la primera entrega.}, para cada $i \in \nat{n}$ es
\begin{align*}
d_p\pi_i\left(\frac{\partial}{\partial \Psi^j}\Big|_p\right) = \delta_{ij} \cdot \frac{\partial}{\partial \varphi_i}\Big|_{p_i}.
\end{align*}

A partir de esto último, afirmamos que $\pi_i^*(d\varphi_i) = d\Psi^i$. Basta probar que en cada punto $p \in U_1 \times \cdots \times U_n$ ambas $1$-formas coinciden en una base de $T_p\T^n$. Tomando los \textit{ganchos} $\{\frac{\partial}{\partial \Psi^i}\big|_p\}$, efectivamente es
\begin{align*}
(\pi_i)_p^*(d\varphi_i)\left(\frac{\partial}{\partial \Psi^j}\Big|_p\right) &= d_{p_i}\varphi_i\left(d_p\pi_i\left(\frac{\partial}{\partial \Psi^j}\Big|_p\right) \right) = d_{p_i}\varphi_i\left(\delta_{ij} \cdot \frac{\partial}{\partial \varphi_i}\Big|_{p_i}\right)
\\&= \delta_{ij}d_{p_i}\varphi_i\left(\frac{\partial}{\partial \varphi_i}\Big|_{p_i}\right) = \delta_{ij} \cdot \delta_{ij}\\
&= \delta_{ij} = d_p\Psi^i\left(\frac{\partial}{\partial \Psi^j}\Big|_p\right).
\end{align*}

Como $\eta$ es una $1$-forma, para cada $i \in \nat{n}$ existe una función suave $g_i \in C^\infty(\Ss^1)$ tal que $\eta = g_i \cdot d\varphi_i$, y se tiene\footnote{Estas propiedades son parte del ejercicio $(8)$ de la práctica $4$, que elegí resolver para la cuarta entrega.} entonces que $\pi_i^\ast(\eta) = g_i \circ \pi_i \cdot \pi_i^*(d\varphi_i) = g_i \circ \pi_i \cdot d\Psi^i$. Reescribiendo, tenemos una expresión para $\omega$ en terminos de $d\Psi^1, \dots, d\Psi^n$,
\begin{align*}
\omega = \pi^\ast_1(\eta) \wedge \cdots \wedge \pi^\ast_1(\eta) = g_1 \circ \pi_1 \cdot d\Psi^1 \wedge \cdots \wedge g_n \circ \pi_n \cdot d\Psi^n = \prod_{i = 1}^{n}g_i \circ \pi_i \cdot d\Psi^1 \wedge \cdots \wedge d\Psi^n.
\end{align*}

Tanto $\Psi$ como cada carta $\psi_i$ tienen dominio denso cuyo complemento es de medida cero, así que en vista de las anteriores caracterizaciones de $\omega$ y $\eta$ podemos calcular sus integrales como
\begin{align*}
\int_{\T^n}\omega = \int_{\R^n}\prod_{i = 1}^{n}g_i \circ \pi_i \cdot dx_1 \cdots dx_n \quad  \text{ y } \quad \int_{\Ss^1} \eta = \int_{\R}g_i(x)dx.
\end{align*}
para cada $i \in \nat{n}$.

Finalmente usando el teorema de Fubini, es
\begin{align*}
\int_{\T^n}\omega &= \int_{\R^n}\prod_{i = 1}^{n}g_i \circ \pi_i(x_1, \dots, x_n) \cdot dx_1 \cdots dx_n = \int_{\R^n}\prod_{i = 1}^{n}g_i(x_i) \cdot dx_1 \cdots dx_n\\
&= \left(\int_{\R}g_1(x_1)dx_1\right) \cdots \left(\int_{\R}g_n(x_n)dx_n\right)\\
&= \left(\int_{\Ss^1} \eta\right)^n.
\end{align*}
\end{proof}

\begin{exercise}{3} Muestre que cuando $n \geq 2$ el grado de toda función diferenciable $f : \Ss^n \to \T^n$ de la $n$-esfera al $n$-toro es nulo.
\end{exercise}
\begin{proof} Consideremos $\eta \in \Omega^1(\Ss^1)$ una forma de volumen. Definimos entonces
\begin{align*}
\omega = \pi^\ast_1(\eta) \wedge \cdots \wedge \pi^\ast_n(\eta) \in \Omega^n(\T^n)
\end{align*}
con $\pi_i : \T^n \to \Ss^1$ la proyección a la $i$-ésima coordenada. 

Al ser $n \geq 2$ sabemos que $H^1(\Ss^n) = 0$, y por lo tanto para todo $i \in \nat{n}$ resulta $[f^*\pi_i^*(\eta)] = 0 \in H^1(\Ss^n)$. Como $f^* : H^\bullet(\T^n) \to H^\bullet(\Ss^n)$ es un morfismo de álgebras, esto dice que
\begin{align*}
[f^*(\omega)] &= f^*([\omega]) = f^*([\pi^\ast_1(\eta) \wedge \cdots \wedge \pi^\ast_1(\eta)])\\
&= f^*([\pi^\ast_1(\eta)]) \wedge \cdots \wedge f^*([\pi^\ast_1(\eta)])\\
&= [0] \wedge \cdots \wedge [0] = [0].
\end{align*}

Vemos así que $f^*(\omega)$ es exacta y por lo tanto, existe $\zeta \in \Omega^{n-1}(\T^n)$ tal que $f^*(\omega) = d\zeta$. Por el teorema de Stokes, esto implica que
\begin{align*}
\int_{\Ss^n}f^*(\omega) = \int_{\Ss^n}d\zeta = \int_{\partial \Ss^n}i^*(\zeta) = 0
\end{align*}
ya que $\Ss^n$ no tiene borde.

Del $\tpaing{Lema $1$}$ y la definición de grado, obtenemos
\begin{align*}
0 = \int_{\Ss^n}f^*(\omega) = \deg(f) \cdot \int_{\T^n}\omega = \deg(f) \cdot \left(\int_{\Ss^1} \eta\right)^n.
\end{align*}
Al ser $\eta$ una forma de volumen de $\Ss^1$, su integral es no nula, y consecuentemente es $\deg(f) = 0$. 
\end{proof}

\paintline

\begin{theorem}{2 (dualidad de Poincaré)} Sea $M$ una variedad conexa y orientable de dimensión $n$. Entonces, para cada $k \in \nat{n}$ se tiene que $H^k(M) \simeq H^{n-k}_c(M)^*$.
\end{theorem}

\begin{exercise}{4} Sea $M$ una variedad compacta, orientable, conexa de dimensión $n$. Sabemos que la cohomología de De Rham de $M$ tiene entonces dimensión total finita, y podemos en consecuencia considerar el entero
\begin{align*}
\chi(M) = \sum_{i=0}^{n}(-1)^i \dim H^i(M)
\end{align*}
al que llamamos la \textit{característica de Euler} de $M$.
\begin{itemize}[listparindent = \parindent]
\item[(a)] Si la dimensión $n$ de $M$ es impar, entonces $\chi(M) = 0$.
\item[(b)] Si la dimensión $n$ de $M$ es par y la de $H^{n/2}(M)$ es par, entonces $\chi(M)$ es un entero par.
\end{itemize}
\end{exercise}
\begin{proof} Notemos que como la variedad $M$ es compacta, su cohomología a soporte compacto coincide con la cohomología de de Rham «a secas». Además, dado que la cohomología de una variedad compacta es finitamente generada, se tiene que
\begin{align*}
H^{n-i}(M) \simeq H^{n-i}(M)^* = H^{n-i}_c(M)^* 
\end{align*}
para cada $i \in \natzero{n}$.

Por el $\tpaing{Teorema $2$}$, tenemos entonces que $H^i(M) \simeq H^{n-i}(M)$ para todo $i \in \natzero{n}$. En particular, si notamos $\beta_i := \dim H^i(M)$ a la dimensión del $i$-ésimo grupo de cohomología, debe ser $\beta_i = \beta_{n-i}$.
Ahora,
\begin{itemize}[listparindent = \parindent]
\item[(a)] Si $n$ es impar, entonces
\begin{align*}
2\chi(M) &= \sum_{i=0}^{n}(-1)^i\beta_i + \sum_{i=0}^{n}(-1)^i\beta_i =
\sum_{i=0}^{n}(-1)^i\beta_i + \sum_{i=0}^{n}(-1)^{n-i}\beta_{n-i}\\
&= \sum_{i=0}^{n}(-1)^i\beta_i + (-1)^n\sum_{i=0}^{n}(-1)^{-i}\beta_{n-i}\\
&= \sum_{i=0}^{n}(-1)^i\beta_i + (-1)^n\sum_{i=0}^{n}(-1)^{i}\beta_{i}\\
&= \chi(M)(1 + (-1)^n) = \chi(M) \cdot  0 = 0.
\end{align*}
Por lo tanto, es $\chi(M) = 0$.
\item[(b)] De una forma similar, si $n$ y es par y $\beta_{n/2}$ también, entonces
\begin{align*}
\chi(M) &= \sum_{i=0}^{n}(-1)^i\beta_i = \sum_{i=0}^{n/2-1}(-1)^i\beta_i + \beta_{n/2} + \sum_{i=n/2+1}^{n}(-1)^i\beta_i\\
&= \sum_{i=0}^{n/2-1}(-1)^i\beta_i + \beta_{n/2} + \sum_{i=0}^{n/2-1}(-1)^{n-i}\beta_{n-i}\\
&= 2\sum_{i=0}^{n/2-1}(-1)^i\beta_i + \beta_{n/2} \equiv \beta_{n/2} \equiv 0 \pmod{2}.
\end{align*}
En consecuencia, la característica de Euler de $M$ es par.
\end{itemize}
\end{proof}


\paintline

\begin{exercise}{5} Sea $G$ un grupo de Lie de dimensión $n$, sea $\mathfrak{g} = T_eG$ su álgebra de Lie y fijemos un producto interno $g_e : \mathfrak{g} \times \mathfrak{g} \to \R$ sobre $\mathfrak{g}$.
\begin{itemize}[listparindent = \parindent]
\item[(a)] Hay una única métrica riemanniana $g$ sobre $G$ que es invariante a izquierda y cuyo valor en $e \in G$ es $g_e$.
\item[(b)] Sea $\mathscr{B} = \{v_1, \dots, v_n\}$ una base de $\mathfrak{g}$ y sean $X_1, \dots, X_n$ los campos tangentes a $G$ invariantes a izquierda que extienden a los elementos de $\mathscr{B}$. Muestre que para cada $i,j$, es \textit{constante} la función $g_{i,j} = g(X_i,X_j)$.

Sabemos (porque el corchete de Lie de campos invariantes a izquierda es él mismo invariante a izquierda) que existen constantes $c_{i,j}^k$ tales que
\begin{align*}
[X_i,X_j] = \sum_{k}c_{i,j}^kX_k.
\end{align*}
Calcule en términos de los escalares $c_{i,j}^k$ y $g_{i,j}$ los símbolos de Christoffel $\Gamma_{ij}^k$ de la conexión de Levi-Civita de G con respecto a los campos $X_1, \dots, X_n$, de manera que se tenga
\begin{align*}
\nabla_{X_i}X_j = \sum_k\Gamma_{ij}^kX_k.
\end{align*}
\item[(c)] Sea $G = \R_{>0 } \times \R$ el grupo de Lie con producto dado por
\begin{align*}
(a,b) \cdot (c,d) = (ac,ad+b)
\end{align*}
para cada $(a,b),(c,d) \in G$, de manera que $G$ es isomorfo de la forma evidente al grupo de matrices 
\begin{align*}
\left\{\begin{pmatrix}
a & b \\
0 & 1
\end{pmatrix} : a >0, b \in \R\right\}.
\end{align*}
El elemento neutro $G$ es $e = (1,0)$ y su álgebra de Lie $\mathfrak{g} = T_eG$ se identifica de manera natural (porque $G$ es un abierto de $\R^2$) con $\R^2$. Dotemos a $G$ de su única métrica invariante a izquierda que en $T_eG$ restringe al producto interno usual de $\R^2$. Encuentre todas las geodésicas que pasan por $e$ que pueda.

Calcule (las componentes en una carta del) tensor de curvatura $R(X,Y)Z$ sobre $G$ y la \textit{curvatura escalar}
\begin{align*}
K(p) = \frac{1}{n(n-1)} \sum_{1 \leq i,j \leq n}g(R(z_i,z_j)z_i,z_j)
\end{align*}
para cada $p \in G$, con $\{z_1, \dots, z_n\}$ una base ortonormal de $T_pG$.
\end{itemize}
\end{exercise}
\begin{proof}
content...
\end{proof}

\paintline

\begin{exercise}{6} Sea $M$ una variedad compacta y orientable de dimensión $4k$.
\begin{itemize}[listparindent = \parindent]
\item[(a)] Muestre que hay una función bilineal no degenerada y simétrica
\begin{align*}
\beta : H^{2k}(M) \times H^{2k}(M) \to \R
\end{align*}
tal que si $\omega$ y $\eta$ son elementos cerrados de $\Omega^{2k}(M)$ entonces 
\begin{align*}
\beta([\omega],[\eta]) = \int_M \omega \wedge \eta.
\end{align*}
Llamamos la signatura de la forma bilineal $\sigma$ la signatura de $M$.
\item[(b)] Determine la signatura de $\Ss^4$, de $\Ss^2 \times \Ss^2$, del toro $T^4$, del espacio proyectivo $P_{\C}^2$ y el producto $P_{\C}^2 \times P_{\C}^2$.
\end{itemize}
\end{exercise}
\begin{proof}
content...
\end{proof}

\paintline

\begin{exercise}{7} Sea $M \subseteq \R^3$ una superficie orientada y sin borde dotada de su métrica riemanniana inducida por la de $\R^3$ y supongamos que hay un campo vectorial $Z \in \X(M)$ sobre $M$ que no se anula en ningún punto.
\begin{itemize}[listparindent = \parindent]
\item[(a)] Muestre que existe una única forma de elegir campos $X,Y \in \X(M)$ tales que para cada $p \in M$ se tiene que $(X_p,Y_p)$ es una base ortonormal positiva de $T_pM$ y $Z_p = \|Z_p\|X_p$. 
\item[(b)] Hay $1$-formas $\alpha,\beta\in \Omega^1(M)$ tales que $\alpha(X) = \beta(Y) = 1$ y $\alpha(Y) = \beta(X) = 0$. Más aún, existe una forma $\eta \in \Omega^1(M)$ tal que
\begin{align*}
d\alpha = \eta \wedge \beta, \quad d\beta = -\eta \wedge \alpha.
\end{align*}
La forma $\sigma  = \alpha \wedge \beta$ no depende de la elección de $Z$, es una forma de volumen sobre $M$ que determina su orientación y es, de hecho, la forma de volumen riemanniano sobre $M$.
\item[(c)] Existe una función diferenciable $K : M \to \R$ tal que
\begin{align*}
d\eta = -K \cdot \sigma
\end{align*}
y esta función no depende de la elección del campo $Z$.
\item[(d)] Si $M$ es compacta, entonces $\int_M K \cdot \sigma = 0$.
\item[(e)] Usando los resultados anteriores, muestre que no hay sobre $\Ss^2$ un campo vectorial tangente que no se anula en ningún punto.
\end{itemize}
\end{exercise}
\begin{proof}
content...
\end{proof}

\end{document}
