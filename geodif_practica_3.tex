\documentclass[11pt]{article}
\usepackage[margin=1in]{geometry} 
\usepackage{amsmath,amsthm,amssymb,amsfonts}

\usepackage{mathpazo}
\usepackage{euler}
\usepackage{xcolor}
\usepackage{tikz}
\usepackage{tikz-cd}
\usetikzlibrary{arrows}
\usetikzlibrary{matrix}
\usepackage{fancyhdr}
\pagestyle{fancy}

\newcommand{\N}{\mathbb{N}}
\newcommand{\Z}{\mathbb{Z}}
\newcommand{\Q}{\mathbb{Q}}
\newcommand{\R}{\mathbb{R}}
\newcommand{\C}{\mathbb{C}}
\newcommand{\Ss}{\mathbb{S}}
\newcommand{\M}[2]{\mathsf{M}_{#1}#2}
\newcommand{\im}{\operatorname{im}}
\newcommand{\eps}{\varepsilon}
\newcommand{\dpart}[2]{\frac{\partial#1}{\partial#2}}
\newcommand{\nat}[1]{[\![#1]\!]}
\newcommand{\natzero}[1]{\nat{#1}_0}
\newcommand{\adj}[1]{\operatorname{adj}(#1)}
\newcommand{\ip}[2]{\langle #1 , #2 \rangle}
\newcommand{\ol}{\overline}
\newcommand{\hook}[3]{\frac{\partial}{\partial x_{#1}}\Big\rvert_{#2}^{#3}}
\usepackage{rotating}
\newcommand*{\isoarrow}[1]{\arrow[#1,"\rotatebox{90}{\LARGE{\(\sim\)}}"
]}
\definecolor{color}{RGB}{155, 132, 0}
\newcommand{\paint}[1]{\color{color}{#1}}

\renewcommand*{\proofname}{\paint{Demostraci\'on}}
\newenvironment{theorem}[2][Teorema]{\begin{trivlist}
\item[\hskip \labelsep \paint{{\bfseries #1}}\hskip \labelsep {\bfseries #2.}]}{\end{trivlist}}
\newenvironment{lemma}[2][Lema]{\begin{trivlist}
\item[\hskip \labelsep \paint{{\bfseries #1}}\hskip \labelsep {\bfseries #2.}]}{\end{trivlist}}
\newenvironment{exercise}[2][Ejercicio]{\begin{trivlist}
\item[\hskip \labelsep \paint{{\bfseries #1}}\hskip \labelsep {\bfseries #2.}]}{\end{trivlist}}
\newenvironment{obs}[2][Observaci\'on]{\begin{trivlist}
\item[\hskip \labelsep \paint{{\bfseries #1.}}]}{\end{trivlist}}
\newenvironment{reflection}[2][Resoluci\']{\begin{trivlist}
\item[\hskip \labelsep {\bfseries #1}\hskip \labelsep {\bfseries #2.}]}{\end{trivlist}}
\newenvironment{proposition}[2][Proposici\'on]{\begin{trivlist}
\item[\hskip \labelsep \paint{{\bfseries #1}}\hskip \labelsep {\bfseries #2.}]}{\end{trivlist}}
\newenvironment{corollary}[2][Corolario]{\begin{trivlist}
\item[\hskip \labelsep {\bfseries #1}\hskip \labelsep {\bfseries #2.}]}{\end{trivlist}}

%-----------------------

\title{
\LARGE{\paint{Geometr\'ia Diferencial}}
\\
\vspace{0.5pt}
\small{\paint{Ejercicios para Entregar - Pr\'actica 3}}
}
\author{\paint{Guido Arnone}}
\date{}
\lhead{Guido Arnone}
\rhead{Pr\'actica 3}

\begin{document}

\maketitle

\begin{center}
\paint{\large{Sobre los Ejercicios}}
\end{center}
\begin{center}
Elej\'i resolver los ejercicios $\paint{(2)}$, $\paint{(8)}$ y $\paint{(9)}$.
\end{center}
\begin{center}
$\paint{
\rule{400pt}{0.5pt}
}$
\vspace{35pt}
\end{center}

Recuerdo primero el siguiente resultado,

\begin{obs}{1} Sea $M$ una variedad y $v \in T_pM$ una derivaci\'on en un punto $p \in M$. Entonces, existe una curva suave $c : (-\eps,\eps) \to M$ tal que $c(0) = p$ y $c'(0) = v$.

En efecto, consideremos primero una carta $(U,\phi)$ con $p \in U \subset \R^n$. Componiendo con una traslaci\'on (que es un difeomorfismo de $\R^n$) si es necesario, podemos suponer que $\phi(p) = 0$. Ahora, como los \emph{ganchos} de $\phi$ en $p$ son una base para $T_pM$, existen \'unicos coeficientes $a_1, \dots, a_n \in \R$ tales que $v =\sum_{i=1}^na_i \frac{\partial}{\partial\phi^i}|_p$. 

Ahora, tomando $\eps > 0$ suficientemente peque\~{n}o como para que $B_\eps(0) \subset \phi(U)$, afirmo que la curva $c : t \in (-\eps,\eps) \mapsto \phi^{-1}(tv) \in M$ cumple lo pedido. En primer lugar tenemos que $c(0) = \phi^{-1}(0) = p$. Observemos tambi\'en que $c$ es suave, pues es la composici\'on de $\phi^{-1}$ que es suave (pues $\phi$ es una carta de $M$) y la curva $\gamma : t \in \R \mapsto t(a_1,\dots,a_n) \in \R^n$ que tambi\'en lo es. 

Por \'ultimo si $g \in C^\infty(M)$, entonces
\begin{align*}
d_0c\left(\frac{d}{dt}\Big|_0\right)(g) &= \frac{d}{dt}\Big|_0(gc) = \frac{d}{dt}\Big|_0(g\phi^{-1}\gamma) = \sum_{i=1}^n\frac{\partial g\phi^{-1}}{\partial x_i}\Big|_{c(0)} \cdot \gamma'_i(0)\\
&= \sum_{i=1}^n\frac{\partial g\phi^{-1}}{\partial x_i}\Big|_{p} \cdot a_i = \sum_{i=1}^na_i\frac{\partial }{\partial \phi^i}\Big|_{p}(g)\\
&= \left(\sum_{i=1}^na_i\frac{\partial }{\partial \phi^i}\Big|_{p}\right)(g) = v(g).
\end{align*}
de forma que $c'(0) = v$.
\end{obs}

\begin{exercise}{2} Sean $M$ una variedad y $f\in C^\infty(M)$. Si $f$ tiene un m\'aximo
local en $p\in M$, entonces $d_pf = 0$.
\end{exercise}
\begin{proof} Como $p$ es un m\'aximo local de $f$, existe un abierto $U \ni p$ tal que $f(q) \leq f(p)$ para cada $q \in U$. Fijemos $v \in T_pM$. Por la observaci\'on anterior, tenemos una curva $c : (-\eps,\eps) \to M$ tal que $c(0) = p$ y $c'(0) = v$. Adem\'as por como construimos la curva en la observaci\'on anterior, tomando un carta $(V,\phi)$ con $p \in V \subset U$ podemos m\'as a\'un suponer que $\im c \subset U$. En consecuencia es $fc(t) \leq f(0) = f(p)$ para cada $t \in (-\eps,\eps)$. Esto es, $0$ resulta un m\'aximo local de la curva suave $fc : (-\eps,\eps) \to \R$ y entonces $(fc)'(0) = 0$. Esto dice que
\begin{align*}
0 = d_0(f\circ c)\left(\frac{d}{dt}\Big|_0\right)  = d_pf\left(d_0c\left(\frac{d}{dt}\Big|_0\right)\right) = d_pf(c'(0)) = d_pf(v).
\end{align*}
Como $d_pf(v) = 0$ para cada $v \in T_pM$, en efecto $d_pf = 0$.
\end{proof}

Probamos ahora la sugerencia del ejercicio $\paint{(8)}$.

\begin{proposition}{2} Sea $G$ un grupo de Lie, $\mathfrak{g}$ su \'algebra de Lie y $X \in \mathfrak{g}$ un campo vectorial invariante a izquierda. Si $g, h \in G$ y $\gamma : (a,b) \to G$ es una curva integral de $X$ que arranca en $g = \gamma(0)$ entonces la curva $\eta : t \in (a,b) \to h\gamma(t) \in G$ es una curva integral de $X$ que arranca en $hg$. 
\end{proposition}
\begin{proof} Como $\eta(0) = h\gamma(0) = hg$, la curva $\eta$ comienza en $hg$. Resta ver que es una curva integral. Fijemos ahora $s \in (a,b)$. Observando que por definici\'on $\eta = L_h \circ \gamma$, es
\begin{align*}
d_s \eta \left(\frac{d}{dt}\Bigg|_{s}\right) &= (d_{\gamma(s)}L_h \circ d_s\gamma) \left(\frac{d}{dt}\Bigg|_{s}\right) = d_{\gamma(s)}L_h \left( d_s\gamma \left(\frac{d}{dt}\Bigg|_{s}\right) \right)\\
& = d_{\gamma(s)}L_h(X_{\gamma(s)}) \stackrel{(X \in \mathfrak{g})}{=} X_{h\gamma(s)} = X_{\eta(s)}.
\end{align*}
Esto es precisamente que $\eta$ sea integral.
\end{proof}

\begin{exercise}{8} Sea $G$ un grupo de Lie, $\mathfrak{g}$ su \'algebra de Lie y $X\in\mathfrak{g}$ un campo vectorial invariante a izquierda. Pruebe que $X$ es \emph{completo} y describa el flujo asociado.
\end{exercise}
\begin{proof} Notemos que por la proposici\'on anterior, alcanza probar que existe una curva integral definida en toda la recta $\gamma : \R \to G$ tal que $\gamma(0) = e$. En tal caso, para cada $h \in G$ la curva $h \gamma : \R \to G$ resultar\'a integral, definida en toda la recta, y comenzar\'a en $h\gamma(0) = he = h$, probando que $X$ es completo.  $\paint{\text{[FALTA VER QUE HAY UNA CURVA INTEGRAL POR LA IDENTIDAD]}}$
\end{proof}

A continuaci\'on, pruebo un resultado auxiliar para el ejercicio siguiente.

\begin{lemma}{3} Si $v\in T_eG$ es un vector tangente a $G$ en $e$ y $X^v \in\mathfrak{g}$ es el \'unico campo vectorial invariante a izquierda tal que $X^v_e=v$, sea $\gamma_v:\R\to G$ la \'unica curva integral de $X$ tal que $\gamma_v(0)=e.$ Entonces, para cada $t,s \in R$ es $\gamma_{tv}(s) = \gamma_{v}(ts)$.
\end{lemma}
\begin{proof} Fijtemos $t \in \R$. Notemos que $tX$ resulta un campo invariante a izquierda pues para cada $g,h \in G$ tenemos $d_hL_g(tX_h) = td_hL_g(X_h) = tX_h$. Como adem\'as es $tX_e = tv$, concluimos as\'i que $X^{tv} = tX^v$.

Ahora, por unicidad de las curvas integrales maximales, para ver que $\eta(s) := \gamma_v(t \cdot s) = \gamma_{tv}(s)$ alcanza probar que $\eta$ es una curva integral de $X^{tv}$ que comienza en $e$. Esto \'ultimo es claro pues $\gamma_v(t \cdot 0) = \gamma_v(0) = e$. Para terminar, veamos que $\eta$ es integral. Si notamos $l_t : s \in \R \mapsto ts \in \R$ entonces $\eta = \gamma_{v} \circ l_t$. Finalmente, 
\begin{align*}
d_s\eta\left(\frac{d}{dt}\Big|_s\right) = (d_{l_t(s)}\gamma_{v} \circ d_sl_t) \left(\frac{d}{dt}\Big|_s\right) = d_{ts}\gamma_{v} \circ l_t'(s) ARREGLAR CUENTA
\end{align*}
\end{proof}

\begin{exercise}{9} Sea $G$ un grupo de Lie, $e$ su elemento neutro y $\mathfrak{g}$ su \'algebra de Lie. Probar que:
\begin{itemize}
\item[a)] Si $v\in T_eG$ es un vector tangente a $G$ en $e$ y $X\in\mathfrak{g}$ es el \'unico campo vectorial invariante a izquierda tal que $X_e=v$, sea $\gamma_v:\R\to G$ la \'unica curva integral de $X$ tal que $\gamma_v(0)=e$. Entonces $\gamma_v$ es un homomorfismo de grupos, esto es,
  \[
  \gamma_v(t+s)=\gamma_v(t)\cdot\gamma_v(s), \qquad\forall t,s\in\R.
  \]

\item[b)] Definimos una funci\'on $\exp:T_eG\to G$ poniendo, para cada $v\in
T_eG$, 
  \[
  \exp(v)=\gamma_v(1).
  \]
Determine la diferencial $\exp_{*0}:T_eG\to T_eG$ y muestre que $\exp$ es
localmente un difeomorfismo alrededor de~$0$.

\item[c)] Muestre que si $v$,~$w\in T_eG$ son tales que $[v,w]=0$, entonces
  \[
  \exp(v+w)=\exp(v)\cdot\exp(w).
  \]
\end{itemize}
\end{exercise}
\begin{proof} Hacemos cada inciso por separado.
\begin{itemize}
\item[a)] Fijemos $t \in \R$ y sea $g = \gamma_v(t)$. Por la $\paint{\text{Proposici\'on $2$}}$, sabemos que
\begin{align*}
\eta(s) := g \gamma_v(s) = \gamma_v(t) \cdot \gamma_v(s)
\end{align*}
es la curva integral maximal que comienza en $g\cdot e  = g$. Por lo tanto, para probar la igualdad del enunciado, alcanza con mostrar que la curva $\xi : s \in \R \mapsto \gamma_v(t+s) \in G$ comienza en $g$ y es integral. En tal caso, como \'esta tambi\'en es una curva definida en todo $\R$, por unicidad de las curvas integrales maximales deber\'a coincidir con $\eta$. 

En primer lugar, notemos que $\xi$ comienza efectivamente en $g$ ya que
\begin{align*}
\xi(0) = \gamma_v(t+0) = \gamma_v(t) = g.
\end{align*}
Si notamos ahora $T_t : s \in \R \mapsto t+s \in \R$ y tomamos $h \in C^\infty(\R)$, para cada $s_0 \in \R$ es
\begin{align*}
d_{s_0}T_t\Big(\frac{d}{ds}\Big|_{s_0}\Big)(h) = \frac{dT_th}{ds}\Big|_{s_0} = h'(t+s_0) =  \frac{d}{ds}\Big|_{t+s_0}(h).
\end{align*}
Como $\xi = \gamma_v \circ T_t$, obtenemos finalmente que
\begin{align*}
d_{s_0}\xi\Big(\frac{d}{ds}\Big|_{s_0}\Big) &= (d_{s_0+t}\gamma_v  \circ d_{s_0} T_t)\Big(\frac{d}{ds}\Big|_{s_0}\Big)= d_{s_0+t}\gamma_v\Big(\frac{d}{ds}\Big|_{s_0+t}\Big)\\
&= \gamma_v'(s_0+t) = X_{\gamma_v(s_0+t)} = X_{\xi(s_0)}.
\end{align*}
Esto termina de probar que $\xi$ es integral, y por tanto la igualdad.
\item[b)] Veamos primero que $\exp$ es diferenciable. $\paint{\text{FALTA VER DIFERENCIABILIDAD, EN CERO AL MENOS}}$.

Ahora, sea $v \in T_eG$ y calculemos $\exp_{\ast 0}(v)$. Consideremos la curva $c : t \in \R \mapsto tv \in T_eG$ que satisface $c(0) = 0$ y $c'(0) = v$. Por el $\paint{\text{Lema $3$}}$, es
\begin{align*}
\exp(c(t)) = \exp(tv) = \gamma_{tv}(1) = \gamma_{v}(t \cdot 1)
\end{align*}
para cada $t \in \R$, de forma que
\begin{align*}
\exp_{*0}(v) = d_0\exp \circ c = d_0\gamma = v 
\end{align*}
y entonces $\exp_{*0} \equiv id_{T_eG}$. 

Por \'ultimo, como el diferencial de la exponencial en $0$ resulta inversible, por el teorema de la funci\'on inversa $\exp$ es un difeomorfismo local alrededor de $0$.
\end{itemize}
\end{proof}

\end{document}
