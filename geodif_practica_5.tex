\documentclass[11pt]{article}
\usepackage[margin=1in]{geometry} 
\usepackage{amsmath,amsthm,amssymb,amsfonts}
\usepackage[utf8]{inputenc}
\usepackage[T1]{fontenc}
\usepackage{microtype}
\usepackage{mathpazo}
\usepackage{euler}
\usepackage{xcolor}
\usepackage{tikz}
\usepackage{tikz-cd}
\usetikzlibrary{arrows}
\usetikzlibrary{matrix}
\usepackage{fancyhdr}
\pagestyle{fancy}
\usepackage{enumitem}

\newcommand{\N}{\mathbb{N}}
\newcommand{\Z}{\mathbb{Z}}
\newcommand{\Q}{\mathbb{Q}}
\newcommand{\R}{\mathbb{R}}
\newcommand{\C}{\mathbb{C}}
\newcommand{\Ss}{\mathbb{S}}
\newcommand{\M}[2]{\mathsf{M}_{#1}#2}
\newcommand{\im}{\operatorname{im}}
\newcommand{\eps}{\varepsilon}
\newcommand{\dpart}[2]{\frac{\partial#1}{\partial#2}}
\newcommand{\nat}[1]{[\![#1]\!]}
\newcommand{\natzero}[1]{\nat{#1}_0}
\newcommand{\adj}[1]{\operatorname{adj}(#1)}
\newcommand{\ip}[2]{\langle #1 , #2 \rangle}
\newcommand{\ol}{\overline}
\newcommand{\hook}[3]{\frac{\partial}{\partial x_{#1}}\Big\rvert_{#2}^{#3}}
\usepackage{rotating}
\newcommand*{\isoarrow}[1]{\arrow[#1,"\rotatebox{90}{\LARGE{\(\sim\)}}"
]}
\renewcommand{\d}{\operatorname{d}}
\definecolor{color}{RGB}{191, 0, 124}
\newcommand{\paint}[1]{\color{color}{#1}}

\renewcommand*{\proofname}{\paint{Demostraci\'on}}
\newenvironment{theorem}[2][Teorema]{\begin{trivlist}
\item[\hskip \labelsep \paint{{\bfseries #1}}\hskip \labelsep {\bfseries #2.}]}{\end{trivlist}}
\newenvironment{lemma}[2][Lema]{\begin{trivlist}
\item[\hskip \labelsep \paint{{\bfseries #1}}\hskip \labelsep {\bfseries #2.}]}{\end{trivlist}}
\newenvironment{exercise}[2][Ejercicio]{\begin{trivlist}
\item[\hskip \labelsep \paint{{\bfseries #1}}\hskip \labelsep {\bfseries #2.}]}{\end{trivlist}}
\newenvironment{obs}[2][Observaci\'on]{\begin{trivlist}
\item[\hskip \labelsep \paint{{\bfseries #1.}}]}{\end{trivlist}}
\newenvironment{reflection}[2][Resoluci\']{\begin{trivlist}
\item[\hskip \labelsep {\bfseries #1}\hskip \labelsep {\bfseries #2.}]}{\end{trivlist}}
\newenvironment{proposition}[2][Proposici\'on]{\begin{trivlist}
\item[\hskip \labelsep \paint{{\bfseries #1}}\hskip \labelsep {\bfseries #2.}]}{\end{trivlist}}
\newenvironment{corollary}[2][Corolario]{\begin{trivlist}
\item[\hskip \labelsep {\bfseries #1}\hskip \labelsep {\bfseries #2.}]}{\end{trivlist}}

%-----------------------

\title{
\LARGE{\paint{Geometr\'ia Diferencial}}
\\
\vspace{0.5pt}
\small{\paint{Ejercicios para Entregar - Pr\'actica 4}}
}
\author{\paint{Guido Arnone}}
\date{}
\lhead{Guido Arnone}
\rhead{Pr\'actica 4}

\begin{document}

\maketitle

\begin{center}
\paint{\large{Sobre los Ejercicios}}
\end{center}
\begin{center}
Elegí los ejercicios $\paint{(6)}$, $\paint{(8)}$ y el inciso $\paint{(d)}$ del ejercicio $\paint{(11)}$.
\end{center}
\begin{center}
$\paint{
\rule{400pt}{0.5pt}
}$
\vspace{35pt}
\end{center}

\begin{exercise}{6} Si $M$ y $N$ son variedades no vac\'i­as, entonces $M\times N$ es
orientable si y s\'olo si $M$ y $N$ lo son.
\end{exercise}
\begin{proof} Veamos ambas implicaciones
\begin{itemize}[listparindent = \parindent]
\item[($\Rightarrow$)] Supongamos que $M \times N$ es orientable, y veamos que $M$ lo es. Un argumento similar prueba lo mismo para $N$.

Sea $(\psi, V)$ una carta de $N$. Como $M \times N$ es orientable, así lo es $M \times V$ pues tomando una forma de volumen en $M \times N$ y restringiéndola a $M \times V$ obtenemos nuevamente una forma de volumen. Dado que $\psi$ es una carta de $N$, ésta es un difeomorfismo de $V$ a $\R^n$ y por lo tanto $id_M \times \psi : M \times V \to M \times \R^n$ es un difeomorfismo de $M \times V$ a $M \times \R^n$, con inversa $id_M \times \psi^{-1}$. En particular de aquí vemos que $M \times \R^n$ es orientable.

En vista de esto, basta probar que si $M$ es una variedad y $M \times \R^n$ es orientable entonces $M$ es orientable. De hecho alcanza probar lo anterior para $n = 1$, pues en tal caso esto nos permitirá probar que de ser $M \times \R^n \simeq (M \times \R^{n-1}) \times \R$ orientable así lo será $M \times \R^{n-1}$. Inductivamente, obtendremos que $M \times \R$ es orientable, y finalmente por el mismo argumento esto dirá que $M$ lo es.

Veamos para terminar que si $M \times \R$ es orientable, entonces $M$ es orientable. En vista de que $T(M \times \R) \simeq TM \times T\R$, para cada carta $(U \times \R, \varphi \times id)$, notaremos a los \textit{ganchos} como

\begin{align*}
\frac{\partial}{\partial (\varphi \times id)^i}\Bigg|_{(p,q)} = \begin{cases}
\frac{\partial}{\partial \varphi^i}|_{(p,q)} &\text{ si $i \leq n$}\\
\frac{d}{dt}|_{(p,q)} &\text{ si $i = n+1$}
\end{cases}
\end{align*}

Sabemos que existe una forma de volumen $\omega  \in \Omega^{m+1}(M \times \R)$, con cierta expresión local

\begin{align*}
\omega|_U = f_U \cdot \d\varphi^1 \wedge \dots \d\varphi^m \wedge \d t
\end{align*}

para cada carta $\varphi \times id : U \times \R \to \R^{m+1}$ de $M \times \R$. Ahora, definimos $i : p \in M \mapsto (p,0) \in M \times \R$ y consideramos $\eta : M \to Alt^m(M)$ definida en cada carta $(U, \varphi)$ como

\begin{align*}
\eta_p(v_1, \dots, v_m) := \omega_{(p,0)}\left(d_pi(v_1), \dots, d_pi(v_m), \frac{d}{dt}\Big|_{(p,0)}\right).
\end{align*}

Como $\eta$ se define a partir de la expresión local de $\omega$, sabemos que está bien definida y es suave. Más aún como $\omega_{(p,0)}$ es $(m+1)$-multilineal alternada y $d_pi$ es lineal, tenemos que $\eta_p$ es $m$-lineal alternada: esto dice que $\eta$ es una $m$-forma de $M$.

Para concluir que $M$ es orientable resta notar que $\eta$ es de volumen: dado $p \in M$ y una carta $(U,\varphi)$ de $M$ con $U \ni p$, como para cada $i \in \nat{n}$ es $d_pi(\frac{\partial}{\partial \varphi^i}|_p)  = \frac{\partial}{\partial \varphi^i}|_{(p,0)}$ se tiene que

\begin{align*}
\eta_p\left(\frac{\partial}{\partial \varphi^1}\Big|_p, \dots, \frac{\partial}{\partial \varphi^n}\Big|_p\right) &= \omega_{(p,0)}\left(\frac{\partial}{\partial \varphi^1}\Big|_{(p,0)}, \dots, \frac{\partial}{\partial \varphi^n}\Big|_{(p,0)}, \frac{\partial}{\partial t}\Big|_{(p,0)}\right) \neq 0
\end{align*}

pues como $\omega_{(p,0)} \not \equiv 0$, evaluando una base de $T_{(p,0)}M \times \R$ debe dar un valor no nulo.

\item[($\Leftarrow$)] Ahora supongamos que tanto $M$ como $N$ son variedades orientables de dimensión $m$ y $n$ respectivamente, de forma que existen atlas orientables $A$ y $A'$ de $M$ y $N$ respectivamente. Veamos que el atlas
\begin{align*}
\mathcal{A} = \{(\phi \times \psi, U \times V) : (\phi,U) \in A, (\psi,V)\in A'\}
\end{align*}
de $M \times N$ resulta orientable.

Sean $\phi \times \psi : U \times V \to \R^{m+n}$ y $\phi' \times \psi': U' \times V' \to \R^{m+n}$ dos cartas de $\mathcal{A}$ y $(p,q)$ un punto de  $U \times V \cap U' \times V'$. Notando $\pi_i : \R^{m+n} \to \R$ a la proyección a la $i$-ésima coordenada, para cada $i,j \in \nat{n+m}$ es
\begin{align*}
\frac{\partial (\phi' \times \psi')^i}{\partial (\phi \times \psi)^j}\Big|_{(p,q)} &= \frac{\partial }{\partial x_j}\Big|_{(\phi(p),\psi(q))}[(\phi' \times \psi')^i \circ (\phi \times \psi)^{-1}]\\
&= \frac{\partial }{\partial x_j}\Big|_{(\phi(p),\psi(q))}[\pi_i \circ (\phi' \times \psi') \circ \phi^{-1} \times \psi^{-1}]\\
&= \frac{\partial }{\partial x_j}\Big|_{(\phi(p),\psi(q))}[(\phi'\phi^{-1} \times \psi'\psi^{-1})^i] 
\end{align*}
y por tanto,
\begin{align*}
\frac{\partial (\phi' \times \psi')^i}{\partial (\phi \times \psi)^j}\Big|_{(p,q)} &= \begin{cases}
\frac{\partial }{\partial x_j}|_{\phi(p)}(\phi'\phi^{-1})^i&\text{si $1 \leq i \leq n$, $1 \leq j \leq n$ }\\
0 &\text{si $1 \leq i \leq n$, $n+1 \leq j \leq n+m$ }\\
0 &\text{si $n+1 \leq i \leq n+m$, $1 \leq j \leq n$ }\\
\frac{\partial }{\partial x_{j-n}}|_{\psi(q)}(\psi'\psi^{-1})^{i-n}&\text{si $n+1 \leq i \leq n+m$, $n+1 \leq j \leq n+m$ }\\
\end{cases}\\
\end{align*}
Como por definición tenemos que $\frac{\partial }{\partial x_j}\Big|_{\phi(p)}(\phi'\phi^{-1})^i = \frac{\partial {\phi'}^i}{\partial \phi^j}\Big|_p$ y $\frac{\partial }{\partial x_{j-n}}\Big|_{\psi(q)}(\phi'\phi^{-1})^{i-n} = \frac{\partial {\psi'}^{i-n}}{\partial \psi^{j-n}}\Big|_q$, en definitiva resulta
\begin{align*}
\left(\frac{\partial (\phi' \times \psi')^i}{\partial (\phi \times \psi)^j}\Big|_{(p,q)}\right)_{i,j} = \begin{pmatrix}
\left(\frac{\partial {\phi'}^i}{\partial \phi^j}\Big|_p\right)_{i,j} & 0\\
0 & \left(\frac{\partial {\psi'}^{i}}{\partial \psi^{j}}\Big|_q\right)_{i,j}
\end{pmatrix}.
\end{align*}

Finalmente como los atlas $A$ y $A'$ son orientados, las matrices de cambio de coordenadas de las cartas $\phi,\phi'$ y $\psi,\psi'$ tienen determintante positivo. De aquí vemos que
\begin{align*}
\det\left(\frac{\partial (\phi' \times \psi')^i}{\partial (\phi \times \psi)^j}\Big|_{(p,q)}\right)_{i,j} &= \det \begin{pmatrix}
\left(\frac{\partial {\phi'}^i}{\partial \phi^j}\Big|_p\right)_{i,j} & 0\\
0 & \left(\frac{\partial {\psi'}^{i}}{\partial \psi^{j}}\Big|_q\right)_{i,j}
\end{pmatrix}\\
& = \det \left(\frac{\partial {\phi'}^i}{\partial \phi^j}\Big|_p\right)_{i,j} \cdot \det \left(\frac{\partial {\psi'}^{i}}{\partial \psi^{j}}\Big|_q\right)_{i,j} > 0,
\end{align*}
lo que termina de probar que $\mathcal{A}$ es orientado.
\end{itemize}
\end{proof}

\begin{center}
$\paint{
\rule{400pt}{0.5pt}
}$
\vspace{10pt}
\end{center}

\begin{exercise}{8} 
\hspace{1pt}
\begin{itemize}[listparindent = \parindent]
\item[a)] Si $f:M\to N$ es una funci\'on diferenciable entre variedades, probar que los pull-backs $f^*:\Omega^k(N)\to\Omega^k(M)$ son tales que
\begin{itemize}[listparindent = \parindent]
\item[(i)] $f^*(\omega_1+\omega_1) = f^*(\omega_1) + f^*(\omega_2)$
\item[(ii)] $f^*(h\cdot \omega_1) = h\circ f\cdot f^*(\omega_1)$ 
\item[(iii)] $f^*(\omega_1\wedge\omega_2) = f^*(\omega_1)\wedge f^*(\omega_2)$
\end{itemize}
para cada $\omega_1$,~$\omega_2\in\Omega^\bullet(N)$ y $h\in C^\infty(N)$.

\item[b)] Si $U$ y $V$ son abiertos de $\R^n$ y $f:U\to V$ es diferenciable,
entonces
\begin{align*}
  f^*(\d x_i)=\sum_{k=1}^n\frac{\partial f_i}{\partial x_k}\d x_k
\end{align*}
y
\begin{align*}
  f^*(g\cdot\d x_1\wedge\cdots\wedge\d x_n)
        = g\circ f
                \cdot\det\Bigl(\frac{\partial f_i}{\partial x_j}\Bigr)_{i,j}
                \cdot \d x_1\wedge 	\cdots\wedge\d x_n
\end{align*}
para cada $i\in\{1,\dots,n\}$ y cada $g\in C^\infty(V)$.
\end{itemize}
\end{exercise}
\begin{proof} Hacemos cada inciso por separado.
\begin{itemize}[listparindent = \parindent]
\item[(a)] Sean $\omega, \omega_1,\omega_2 \in \Omega^k(N)$ y $\eta \in \Omega^l(N)$. Para verificar las igualdades del enunciado basta ver que, en cada punto de la variedad, las formas coinciden en todo vector tangente. Fijamos entonces $p \in N$ y $v_1, \dots, v_k, v_{k+1}, \dots, v_{k+l} \in T_pN \subset TN$. 

En primer lugar, $\paint{(i)}$ e $\paint{(ii)}$ son ciertas pues
\begin{align*}
f^*(\omega_1+\omega_2)_p(v_1, \dots, v_k) &= (\omega_1+\omega_2)_{f(p)}(f_{\ast,p}(v_1), \dots, f_{\ast,p}(v_k))\\
&= ({\omega_1})_{f(p)}(f_{\ast,p}(v_1), \dots, f_{\ast,p}(v_k)) + ({\omega_2})_{f(p)}(f_{\ast,p}(v_1), \dots, f_{\ast,p}(v_k))\\
& = f^*(\omega_1)_p(v_1, \dots, v_k) + f^*(\omega_2)_p(v_1,\dots,v_k)
\end{align*}
y
\begin{align*}
f^*(h \cdot \omega_1)_p(v_1, \dots, v_k) &= (h \cdot \omega_1)_{f(p)}(f_{\ast,p}(v_1),\dots,f_{\ast,p}(v_k))\\
& = h(f(p)) \cdot {\omega_1}_{f(p)}(f_{\ast,p}(v_1),\dots,f_{\ast,p}(v_k))\\
&= h(f(p)) \cdot f^*(\omega_1)_p(v_1,\dots,v_k).
\end{align*}

Ahora observemos que si tomamos $\sigma \in \Ss_{k+l}$ y notamos $\tilde{v}_i := d_pf(v_i)$, entonces 
\begin{align*}
\sigma \cdot (\omega \otimes \eta)_{f(p)} (d_pf(v_1), \dots d_pf(v_{k+l})) &= (\omega \otimes \eta)_{f(p)} (\tilde{v}_{\sigma(1)}, \dots \tilde{v}_{\sigma(k+l)})\\
&=\omega_{f(p)} (\tilde{v}_{\sigma(1)}, \dots \tilde{v}_{\sigma(k)}) \cdot \eta_{f(p)}(\tilde{v}_{\sigma(k+1)}), \dots, \tilde{v}_{\sigma(k+l)})\\
& = f^*(\omega_p) (v_{\sigma(1)}, \dots v_{\sigma(k)}) \cdot f^*(\eta)_p(v_{\sigma(k+1)}, \dots, v_{\sigma(k+l)})\\
& = \sigma \cdot (f^*(\omega) \otimes f^*(\eta))_p(v_{1},\dots, v_{k+l}).
\end{align*}

En consecuencia se tiene que
\begin{align*}
f^*(\omega \wedge \eta)_p(v_1, \dots, v_{k+l}) &= (\omega \wedge \eta)_{f(p)}(d_pf(v_1), \dots, d_pf(v_{k+l}))\\
&= \frac{1}{k!l!}\sum_{\sigma \in \Ss_{k+l}}(-1)^\sigma \sigma \cdot (\omega \otimes \eta)_{f(p)} (d_pf(v_1), \dots d_pf(v_{k+l}))\\
& = \frac{1}{k!l!}\sum_{\sigma \in \Ss_{k+l}}(-1)^\sigma \sigma \cdot (f^*(\omega) \otimes f^*(\eta))_p(v_{1},\dots, v_{k+l})\\
&= (f^*(\omega) \wedge f^*(\eta))_p (v_{1},\dots, v_{k+l}),
\end{align*}
lo que prueba $\paint{(iii)}$.
\item[b)] Dado $s \in \nat{n}$, para cada $p \in U$ y $g \in C^\infty(V)$ es
\begin{align*}
d_pf\left(\frac{\partial}{\partial x_s}\Big|_p\right)(g) &= \frac{\partial gf}{\partial x_s}\Big|_p = (D_p(gf))_s = (D_{f(p)}gD_pf)_s\\
& = \sum_{j = 1}^n\frac{\partial g}{\partial x_j}\Big|_{f(p)} \cdot \frac{\partial f_j}{\partial x_s}\Big|_p = \sum_{j = 1}^n\frac{\partial f_j}{\partial x_s}\Big|_p\cdot \frac{\partial}{\partial x_j}\Big|_{f(p)}(g).
\end{align*}
y entonces
\begin{align*}
d_pf\left(\frac{\partial}{\partial x_s}\Big|_p\right) = \sum_{j = 1}^n\frac{\partial f_j}{\partial x_s}\Big|_p\cdot \frac{\partial}{\partial x_j}\Big|_{f(p)}.
\end{align*}.

Si ahora tomamos $i \in \nat{n}$, se tiene que
\begin{align*}
f^*(\d x_i)_p\left(\frac{\partial}{\partial x_s}\Big|_p\right) &= d{x_i}_{f(p)}\left(d_pf\left(\frac{\partial}{\partial x_s}\Big|_p\right)\right) = d{x_i}_{f(p)}\left(\sum_{j = 1}^n\frac{\partial f_j}{\partial x_s}\Big|_p\cdot \frac{\partial}{\partial x_j}\Big|_{f(p)}\right)\\
& = \sum_{j = 1}^n\frac{\partial f_j}{\partial x_s}\Big|_p\cdot d{x_i}_{f(p)}\left(\frac{\partial}{\partial x_j}\Big|_{f(p)}\right) = \frac{\partial f_i}{\partial x_s}.
\end{align*}

Como a su vez se satisface
\begin{align*}
\sum_{k=1}^n\frac{\partial f_i}{\partial x_k}(\d {x_k})_p\left(\frac{\partial}{\partial x_s}\Big|_{p}\right) = \sum_{k=1}^n\frac{\partial f_i}{\partial x_k}\delta_{ks} = \frac{\partial f_i}{\partial x_s},
\end{align*}
deber ser 
\begin{align*}
f^*(\d x_i)_p = \sum_{k=1}^n\frac{\partial f_i}{\partial x_k}(\d {x_k})_{p}
\end{align*}
pues ambos lados de la igualdad coinciden en la base de los \textit{ganchos} $\{\frac{\partial}{\partial x_s}|_p\}_{1 \leq s \leq n}$. Dado que esto es cierto para cualquier punto $p \in U$, se tiene
\begin{align}
f^*(\d x_i) = \sum_{k=1}^n\frac{\partial f_i}{\partial x_k}\d {x_k}.
\end{align}

Finalmente, usando $\paint{(a)}$ vemos que
\begin{align*}
f^*(g\cdot\d x_1\wedge\cdots\wedge\d x_n) &= g \circ f \cdot f^*(\d x_1)\wedge\dots\wedge f^*(\d x_n)
\end{align*}
así que para terminar debemos probar que
\begin{align*}
f^*(\d x_1)\wedge\dots\wedge f^*(\d x_n) = \det\left(\frac{\partial f_i}{\partial x_j}\right)_{ij} \cdot  \d x_1 \wedge\dots\wedge\d x_n.
\end{align*}

En efecto, de $\paint{(1)}$ es
\begin{align*}
f^*(\d x_1)\wedge\dots\wedge f^*(\d x_n) &= \sum_{i_1=1}^n\frac{\partial f_1}{\partial x_{i_1}}\d {x_{i_1}} \wedge \dots \wedge \sum_{i_n=1}^n\frac{\partial f_n}{\partial x_{i_n}}\d {x_{i_n}}\\
&= \sum_{i_1, \dots, i_n}\prod_{j=1}^n\frac{\partial f_j}{\partial x_{i_j}} \d x_{i_1} \wedge\dots\wedge \d x_{i_n}\\
&= \sum_{i_1 < \dots < i_n} \prod_{j=1}^n\frac{\partial f_j}{\partial x_{i_j}} \d x_{i_1} \wedge\dots\wedge \d x_{i_n}\\
&= \sum_{\sigma \in \Ss_n} \prod_{j=1}^n\frac{\partial f_j}{\partial x_{\sigma(j)}} \d x_{\sigma(1)} \wedge\dots\wedge \d x_{\sigma(n)}\\
&= \sum_{\sigma \in \Ss_n} (-1)^\sigma \prod_{j=1}^n\frac{\partial f_j}{\partial x_{\sigma(j)}} \d x_1 \wedge\dots\wedge \d x_n\\
&= \left(\sum_{\sigma \in \Ss_n} (-1)^\sigma \prod_{j=1}^n\frac{\partial f_j}{\partial x_{\sigma(j)}}\right) \d x_1 \wedge\dots\wedge \d x_n\\
&= \det\left(\frac{\partial f_i}{\partial x_j}\right)_{ij} \cdot  \d x_1 \wedge\dots\wedge\d x_n.
\end{align*}
\end{itemize}
\end{proof}

\begin{center}
$\paint{
\rule{400pt}{0.5pt}
}$
\vspace{10pt}
\end{center}

\begin{exercise}{11 (d)} Calcule la cohomología de un producto cartesiano de dos esferas.
\end{exercise}
\begin{proof} Sean $n,m \in \N$. Como $\Ss^n$ es una variedad compacta, podemos usar la fórmula de Künneth: para todo $q \geq 0$ es 
\begin{align}
H^q(\Ss^n \times \Ss^m) = \bigoplus_{r+s = q} H^r(\Ss^n) \otimes H^s(\Ss^m).
\end{align}
Sabemos además que la cohomología de la $k$-esfera es $\R$ en los grados $0$ y $k$, y nula en los demás.

Por lo tanto, para que $H^r(\Ss^n \times \Ss^n)$ sea no nulo es condición necesaria que $r\in \{0,n\}$ y $s \in \{0,m\}$. En particular debe ser $r+s \in \{0,n,m,n+m\}$. Al $\Ss^n \times  \Ss^m$ ser arcoconexo, ya sabemos que $H^0(\Ss^n \times \Ss^m) \simeq \R$. 

En los otros casos, usando $\paint{(2)}$ tenemos que
\begin{align*}
H^n(\Ss^n \times \Ss^m) &= H^n(\Ss^n) \otimes H^0(\Ss^m) \oplus H^0(\Ss^n) \otimes H^n(\Ss^m) = \R \otimes \R \oplus \R \otimes \R^{\delta_{nm}} = \R \oplus \R^{\delta_{nm}}
\end{align*}
y similarmente $H^m(\Ss^n \times \Ss^m) = \R^{\delta_{nm}} \oplus \R$. 

Por último, del mismo modo es 
\begin{align*}
H^{n+m}(\Ss^n \times \Ss^m) = H^n(\Ss^n) \otimes H^m(\Ss^m) = \R \otimes \R = \R.
\end{align*}
Por lo tanto, si $n = m$ entonces
\begin{align*}
H^q(\Ss^n \times \Ss^n) = \begin{cases}
\R \quad &\text{si $q = 0$ o $q = 2n$}\\
\R^2 \quad &\text{si $q = n$}\\
0 \quad &\text{en caso contrario}
\end{cases}
\end{align*}
y en los demás casos, es
\begin{align*}
H^q(\Ss^n \times \Ss^m) = \begin{cases}
\R \quad &\text{si $q \in \{0,n,m,n+m\}$}\\
0 \quad &\text{en caso contrario}
\end{cases}
\end{align*}
\end{proof}


\end{document}
